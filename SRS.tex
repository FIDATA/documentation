\section{1. Scope.}

This section shall be divided into the following paragraphs.

\subsection{1.1 Identification.}

This paragraph shall contain a full identification of the system and the
software to which this document applies, including, as applicable,
identification number(s), title(s), abbreviation(s), version number(s),
and release number(s).

\subsection{1.2 System overview.}

This paragraph shall briefly state the purpose of the system and the
software to which this document applies. It shall describe the general
nature of the system and software; summarize the history of system
development, operation, and maintenance; identify the project sponsor,
acquirer, user, developer, and support agencies; identify current and
planned operating sites; and list other relevant documents.

\subsection{1.3 Document overview.}

This paragraph shall summarize the purpose and contents of this document
and shall describe any security or privacy considerations associated
with its use.

\section{2. Referenced documents.}

This section shall list the number, title, revision, and date of all
documents referenced in this specification. This section shall also
identify the source for all documents not available through normal
Government stocking activities.

\section{3. Requirements.}

This section shall be divided into the following paragraphs to specify
the CSCI requirements, that is, those characteristics of the CSCI that
are conditions for its acceptance. CSCI requirements are software
requirements generated to satisfy the system requirements allocated to
this CSCI. Each requirement shall be assigned a project-unique
identifier to support testing and traceability and shall be stated in
such a way that an objective test can be defined for it. Each
requirement shall be annotated with associated qualification method(s)
(see section 4) and traceability to system (or subsystem, if applicable)
requirements (see section 5.a) if not provided in those sections. The
degree of detail to be provided shall be guided by the following rule:
Include those characteristics of the CSCI that are conditions for CSCI
acceptance; defer to design descriptions those characteristics that the
acquirer is willing to leave up to the developer. If there are no
requirements in a given paragraph, the paragraph shall so state. If a
given requirement fits into more than one paragraph, it may be stated
once and referenced from the other paragraphs.

\subsection{3.1 Required states and modes.}

If the CSCI is required to operate in more than one state or mode having
requirements distinct from other states or modes, this paragraph shall
identify and define each state and mode. Examples of states and modes
include: idle, ready, active, post-use analysis, training, degraded,
emergency, backup, wartime, peacetime. The distinction between states
and modes is arbitrary. A CSCI may be described in terms of states only,
modes only, states within modes, modes within states, or any other
scheme that is useful. If no states or modes are required, this
paragraph shall so state, without the need to create artificial
distinctions. If states and/or modes are required, each requirement or
group of requirements in this specification shall be correlated to the
states and modes. The correlation may be indicated by a table or other
method in this paragraph, in an appendix referenced from this paragraph,
or by annotation of the requirements in the paragraphs where they
appear.

\subsection{3.2 CSCI capability requirements.}

This paragraph shall be divided into subparagraphs to itemize the
requirements associated with each capability of the CSCI. A
``capability'' is defined as a group of related requirements. The word
``capability'' may be replaced with ``function,'' ``subject,''
``object,'' or other term useful for presenting the requirements.

\subsubsection{3.2.x (CSCI capability).}

This paragraph shall identify a required CSCI capability and shall
itemize the requirements associated with the capability. If the
capability can be more clearly specified by dividing it into constituent
capabilities, the constituent capabilities shall be specified in
subparagraphs. The requirements shall specify required behavior of the
CSCI and shall include applicable parameters, such as response times,
throughput times, other timing constraints, sequencing, accuracy,
capacities (how much/how many), priorities, continuous operation
requirements, and allowable deviations based on operating conditions.
The requirements shall include, as applicable, required behavior under
unexpected, unallowed, or ``out of bounds'' conditions, requirements for
error handling, and any provisions to be incorporated into the CSCI to
provide continuity of operations in the event of emergencies. Paragraph
3.3.x of this DID provides a list of topics to be considered when
specifying requirements regarding inputs the CSCI must accept and
outputs it must produce.

\subsection{3.3 CSCI external interface requirements.}

This paragraph shall be divided into subparagraphs to specify the
requirements, if any, for the CSCI's external interfaces. This paragraph
may reference one or more Interface Requirements Specifications (IRSs)
or other documents containing these requirements.

\subsubsection{3.3.1 Interface identification and diagrams.}

This paragraph shall identify the required external interfaces of the
CSCI (that is, relationships with other entities that involve sharing,
providing or exchanging data). The identification of each interface
shall include a project-unique identifier and shall designate the
interfacing entities (systems, configuration items, users, etc.) by
name, number, version, and documentation references, as applicable. The
identification shall state which entities have fixed interface
characteristics (and therefore impose interface requirements on
interfacing entities) and which are being developed or modified (thus
having interface requirements imposed on them). One or more interface
diagrams shall be provided to depict the interfaces.

\subsubsection{3.3.x (Project unique identifier of interface).}

This paragraph (beginning with 3.3.2) shall identify a CSCI external
interface by project unique identifier, shall briefly identify the
interfacing entities, and shall be divided into subparagraphs as needed
to state the requirements imposed on the CSCI to achieve the interface.
Interface characteristics of the other entities involved in the
interface shall be stated as assumptions or as ``When {[}the entity not
covered{]} does this, the CSCI shall\ldots{},'' not as requirements on
the other entities. This paragraph may reference other documents (such
as data dictionaries, standards for communication protocols, and
standards for user interfaces) in place of stating the information here.
The requirements shall include the following, as applicable, presented
in any order suited to the requirements, and shall note any differences
in these characteristics from the point of view of the interfacing
entities (such as different expectations about the size, frequency, or
other characteristics of data elements):

Priority that the CSCI must assign the interface

Requirements on the type of interface (such as real-time data transfer,
storage-and-retrieval of data, etc.) to be implemented

Required characteristics of individual data elements that the CSCI must
provide, store, send, access, receive, etc., such as:

Names/identifiers

\begin{enumerate}
\itemsep1pt\parskip0pt\parsep0pt
\item
  Project-unique identifier
\item
  Non-technical (natural language) name
\item
  DoD standard data element name
\item
  Technical name (e.g., record or data structure name in code or
  database)
\item
  Abbreviations or synonymous names
\end{enumerate}

Data type (alphanumeric, integer, etc.)

Size and format (such as length and punctuation of a character string)

Units of measurement (such as meters, dollars, nanoseconds)

Range or enumeration of possible values (such as 0-99)

Accuracy (how correct) and precision (number of significant digits)

Priority, timing, frequency, volume, sequencing, and other constraints,
such as whether the data element may be updated and whether business
rules apply

Security and privacy constraints

Sources (setting/sending entities) and recipients (using/receiving
entities)

Required characteristics of data element assemblies (records, messages,
files, arrays, displays, reports, etc.) that the CSCI must provide,
store, send, access, receive, etc., such as:

\begin{enumerate}
\itemsep1pt\parskip0pt\parsep0pt
\item
  Names/identifiers

  \begin{enumerate}
  \itemsep1pt\parskip0pt\parsep0pt
  \item
    Project-unique identifier
  \item
    Non-technical (natural language) name
  \item
    Technical name (e.g., record or data structure name in code or
    database)
  \item
    Abbreviations or synonymous names
  \end{enumerate}
\item
  Data elements in the assembly and their structure (number, order,
  grouping)
\item
  Medium (such as disk) and structure of data elements/assemblies on the
  medium
\item
  Visual and auditory characteristics of displays and other outputs
  (such as colors, layouts, fonts, icons and other display elements,
  beeps, lights)
\item
  Relationships among assemblies, such as sorting/access characteristics
\item
  Priority, timing, frequency, volume, sequencing, and other
  constraints, such as whether the assembly may be updated and whether
  business rules apply
\item
  Security and privacy constraints
\item
  Security and privacy constraints
\end{enumerate}

Required characteristics of communication methods that the CSCI must use
for the interface, such as:

\begin{enumerate}
\itemsep1pt\parskip0pt\parsep0pt
\item
  Project-unique identifier(s)
\item
  Communication links/bands/frequencies/media and their characteristics
\item
  Message formatting
\item
  Flow control (such as sequence numbering and buffer allocation)
\item
  Data transfer rate, whether periodic/aperiodic, and interval between
  transfers
\item
  Routing, addressing, and naming conventions
\item
  Transmission services, including priority and grade
\item
  Safety/security/privacy considerations, such as encryption, user
  authentication, compartmentalization, and auditing
\end{enumerate}

Required characteristics of protocols the CSCI must use for the
interface, such as:

\begin{enumerate}
\itemsep1pt\parskip0pt\parsep0pt
\item
  Project-unique identifier(s)
\item
  Priority/layer of the protocol
\item
  Packeting, including fragmentation and reassembly, routing, and
  addressing
\item
  Legality checks, error control, and recovery procedures
\item
  Synchronization, including connection establishment, maintenance,
  termination
\item
  Status, identification, and any other reporting features
\end{enumerate}

Other required characteristics, such as physical compatibility of the
interfacing entities (dimensions, tolerances, loads, plug compatibility,
etc.), voltages, etc.

\subsection{3.4 CSCI internal interface requirements.}

This paragraph shall specify the requirements, if any, imposed on
interfaces internal to the CSCI. If all internal interfaces are left to
the design, this fact shall be so stated. If such requirements are to be
imposed, paragraph 3.3 of this DID provides a list of topics to be
considered.

\subsection{3.5 CSCI internal data requirements.}

This paragraph shall specify the requirements, if any, imposed on data
internal to the CSCI. Included shall be requirements, if any, on
databases and data files to be included in the CSCI. If all decisions
about internal data are left to the design, this fact shall be so
stated. If such requirements are to be imposed, paragraphs 3.3.x.c and
3.3.x.d of this DID provide a list of topics to be considered.

\subsection{3.6 Adaptation requirements.}

This paragraph shall specify the requirements, if any, concerning
installation-dependent data to be provided by the CSCI (such as
site-dependent latitude and longitude or site-dependent state tax codes)
and operational parameters that the CSCI is required to use that may
vary according to operational needs (such as parameters indicating
operation-dependent targeting constants or data recording).

\subsection{3.7 Safety requirements.}

This paragraph shall specify the CSCI requirements, if any, concerned
with preventing or minimizing unintended hazards to personnel, property,
and the physical environment. Examples include safeguards the CSCI must
provide to prevent inadvertent actions (such as accidentally issuing an
``auto pilot off'' command) and non-actions (such as failure to issue an
intended ``auto pilot off'' command). This paragraph shall include the
CSCI requirements, if any, regarding nuclear components of the system,
including, as applicable, prevention of inadvertent detonation and
compliance with nuclear safety rules.

\subsection{3.8 Security and privacy requirements.}

This paragraph shall specify the CSCI requirements, if any, concerned
with maintaining security and privacy. These requirements shall include,
as applicable, the security/privacy environment in which the CSCI must
operate, the type and degree of security or privacy to be provided, the
security/privacy risks the CSCI must withstand, required safeguards to
reduce those risks, the security/privacy policy that must be met, the
security/privacy accountability the CSCI must provide, and the criteria
that must be met for security/privacy certification/accreditation.

\subsection{3.9 CSCI environment requirements.}

This paragraph shall specify the requirements, if any, regarding the
environment in which the CSCI must operate. Examples include the
computer hardware and operating system on which the CSCI must run.
(Additional requirements concerning computer resources are given in the
next paragraph.)

\subsection{3.10 Computer resource requirements.}

This paragraph shall be divided into the following subparagraphs.

\subsubsection{3.10.1 Computer hardware requirements.}

This paragraph shall specify the requirements, if any, regarding
computer hardware that must be used by the CSCI. The requirements shall
include, as applicable, number of each type of equipment, type, size,
capacity, and other required characteristics of processors, memory,
input/output devices, auxiliary storage, communications/network
equipment, and other required equipment.

\subsubsection{3.10.2 Computer hardware resource utilization
requirements.}

This paragraph shall specify the requirements, if any, on the CSCI's
computer hardware resource utilization, such as maximum allowable use of
processor capacity, memory capacity, input/output device capacity,
auxiliary storage device capacity, and communications/network equipment
capacity. The requirements (stated, for example, as percentages of the
capacity of each computer hardware resource) shall include the
conditions, if any, under which the resource utilization is to be
measured.

\subsubsection{3.10.3 Computer software requirements.}

This paragraph shall specify the requirements, if any, regarding
computer software that must be used by, or incorporated into, the CSCI.
Examples include operating systems, database management systems,
communications/ network software, utility software, input and equipment
simulators, test software, and manufacturing software. The correct
nomenclature, version, and documentation references of each such
software item shall be provided.

\subsubsection{3.10.4 Computer communications requirements.}

This paragraph shall specify the additional requirements, if any,
concerning the computer communications that must be used by the CSCI.
Examples include geographic locations to be linked; configuration and
network topology; transmission techniques; data transfer rates;
gateways; required system use times; type and volume of data to be
transmitted/received; time boundaries for transmission/
reception/response; peak volumes of data; and diagnostic features.

\subsection{3.11 Software quality factors.}

This paragraph shall specify the CSCI requirements, if any, concerned
with software quality factors identified in the contract or derived from
a higher level specification. Examples include quantitative requirements
regarding CSCI functionality (the ability to perform all required
functions), reliability (the ability to perform with correct, consistent
results), maintainability (the ability to be easily corrected),
availability (the ability to be accessed and operated when needed),
flexibility (the ability to be easily adapted to changing requirements),
portability (the ability to be easily modified for a new environment),
reusability (the ability to be used in multiple applications),
testability (the ability to be easily and thoroughly tested), usability
(the ability to be easily learned and used), and other attributes.

\subsection{3.12 Design and implementation constraints.}

This paragraph shall specify the requirements, if any, that constrain
the design and implementation of the CSCI. These requirements may be
specified by reference to appropriate commercial or military standards
and specifications. Examples include requirements concerning:

\begin{enumerate}
\itemsep1pt\parskip0pt\parsep0pt
\item
  Use of a particular CSCI architecture or requirements on the
  architecture, such as required databases or other software units; use
  of standard, military, or existing components; or use of
  Government/acquirer-furnished property (equipment, information, or
  software)
\item
  Use of particular design or implementation standards; use of
  particular data standards; use of a particular programming language
\item
  Flexibility and expandability that must be provided to support
  anticipated areas of growth or changes in technology, threat, or
  mission
\end{enumerate}

\subsection{3.13 Personnel-related requirements.}

This paragraph shall specify the CSCI requirements, if any, included to
accommodate the number, skill levels, duty cycles, training needs, or
other information about the personnel who will use or support the CSCI.
Examples include requirements for number of simultaneous users and for
built-in help or training features. Also included shall be the human
factors engineering requirements, if any, imposed on the CSCI. These
requirements shall include, as applicable, considerations for the
capabilities and limitations of humans; foreseeable human errors under
both normal and extreme conditions; and specific areas where the effects
of human error would be particularly serious. Examples include
requirements for color and duration of error messages, physical
placement of critical indicators or keys, and use of auditory signals.

\subsection{3.14 Training-related requirements.}

This paragraph shall specify the CSCI requirements, if any, pertaining
to training. Examples include training software to be included in the
CSCI.

\subsection{3.15 Logistics-related requirements.}

This paragraph shall specify the CSCI requirements, if any, concerned
with logistics considerations. These considerations may include: system
maintenance, software support, system transportation modes, supply
system requirements, impact on existing facilities, and impact on
existing equipment.

\subsection{3.16 Other requirements.}

This paragraph shall specify additional CSCI requirements, if any, not
covered in the previous paragraphs.

\subsection{3.17 Packaging requirements.}

This section shall specify the requirements, if any, for packaging,
labeling, and handling the CSCI for delivery (for example, delivery on 8
track magnetic tape labelled and packaged in a certain way). Applicable
military specifications and standards may be referenced if appropriate.

\subsection{3.18 Precedence and criticality of requirements.}

This paragraph shall specify, if applicable, the order of precedence,
criticality, or assigned weights indicating the relative importance of
the requirements in this specification. Examples include identifying
those requirements deemed critical to safety, to security, or to privacy
for purposes of singling them out for special treatment. If all
requirements have equal weight, this paragraph shall so state.

\section{4. Qualification provisions.}

This section shall define a set of qualification methods and shall
specify for each requirement in Section 3 the method(s) to be used to
ensure that the requirement has been met. A table may be used to present
this information, or each requirement in Section 3 may be annotated with
the method(s) to be used. Qualification methods may include:

\begin{enumerate}
\itemsep1pt\parskip0pt\parsep0pt
\item
  Demonstration: The operation of the CSCI, or a part of the CSCI, that
  relies on observable functional operation not requiring the use of
  instrumentation, special test equipment, or subsequent analysis.
\item
  Test: The operation of the CSCI, or a part of the CSCI, using
  instrumentation or other special test equipment to collect data for
  later analysis.
\item
  Analysis: The processing of accumulated data obtained from other
  qualification methods. Examples are reduction, interpretation, or
  extrapolation of test results.
\item
  Inspection: The visual examination of CSCI code, documentation, etc.
\item
  Special qualification methods: Any special qualification methods for
  the CSCI, such as special tools, techniques, procedures, facilities,
  and acceptance limits.
\end{enumerate}

\section{5. Requirements traceability.}

This paragraph shall contain:

\begin{enumerate}
\itemsep1pt\parskip0pt\parsep0pt
\item
  Traceability from each CSCI requirement in this specification to the
  system (or subsystem, if applicable) requirements it addresses.
  (Alternatively, this traceability may be provided by annotating each
  requirement in Section 3.) \\\\ Note: Each level of system refinement
  may result in requirements not directly traceable to higher-level
  requirements. For example, a system architectural design that creates
  multiple CSCIs may result in requirements about how the CSCIs will
  interface, even though these interfaces are not covered in system
  requirements. Such requirements may be traced to a general requirement
  such as ``system implementation'' or to the system design decisions
  that resulted in their generation.
\item
  Traceability from each system (or subsystem, if applicable)
  requirement allocated to this CSCI to the CSCI requirements that
  address it. All system (subsystem) requirements allocated to this CSCI
  shall be accounted for. Those that trace to CSCI requirements
  contained in IRSs shall reference those IRSs.
\end{enumerate}

\section{6. Notes.}

This section shall contain any general information that aids in
understanding this specification (e.g., background information,
glossary, rationale). This section shall include an alphabetical listing
of all acronyms, abbreviations, and their meanings as used in this
document and a list of any terms and definitions needed to understand
this document.

\section{A. Appendixes.}

Appendixes may be used to provide information published separately for
convenience in document maintenance (e.g., charts, classified data). As
applicable, each appendix shall be referenced in the main body of the
document where the data would normally have been provided. Appendixes
may be bound as separate documents for ease in handling. Appendixes
shall be lettered alphabetically (A, B, etc.).
