\section{1. Scope.}

This section shall be divided into the following paragraphs.

\subsection{1.1 Identification.}

This paragraph shall contain a full identification of the system to
which this document applies, including, as applicable, identification
number(s), title(s), abbreviation(s), version number(s), and release
number(s).

\subsection{1.2 System overview.}

This paragraph shall briefly state the purpose of the system to which
this document applies. It shall describe the general nature of the
system; summarize the history of system development, operation, and
maintenance; identify the project sponsor, acquirer, user, developer,
and support agencies; identify current and planned operating sites; and
list other relevant documents.

\subsection{1.3 Document overview.}

This paragraph shall summarize the purpose and contents of this document
and shall describe any security or privacy considerations associated
with its use.

\section{2. Referenced documents.}

This section shall list the number, title, revision, and date of all
documents referenced in this document. This section shall also identify
the source for all documents not available through normal Government
stocking activities.

\section{3. Current system or situation.}

This section shall be divided into the following paragraphs to describe
the system or situation as it currently exists.

\subsection{3.1 Background, objectives, and scope.}

This paragraph shall describe the background, mission or objectives, and
scope of the current system or situation.

\subsection{3.2 Operational policies and constraints.}

This paragraph shall describe any operational policies and constraints
that apply to the current system or situation.

\subsection{3.3 Description of current system or situation.}

This paragraph shall provide a description of the current system or
situation, identifying differences associated with different states or
modes of operation (for example, regular, mainte-nance, training,
degraded, emergency, alternative-site, wartime, peacetime). The
distinction between states and modes is arbitrary. A system may be
described in terms of states only, modes only, states within modes,
modes within states, or any other scheme that is useful. If the system
operates without states or modes, this paragraph shall so state, without
the need to create artificial distinctions. The description shall
include, as applicable:

\begin{enumerate}
\itemsep1pt\parskip0pt\parsep0pt
\item
  The operational environment and its characteristics
\item
  Major system components and the interconnections among these
  components
\item
  Interfaces to external systems or procedures
\item
  Capabilities/functions of the current system
\item
  Charts and accompanying descriptions depicting inputs, outputs, data
  flow, and manual and automated processes sufficient to understand the
  current system or situation from the user's point of view
\item
  Performance characteristics, such as speed, throughput, volume,
  frequency
\item
  Quality attributes, such as reliability, maintainability,
  availability, flexibility, portability, usability, efficiency
\item
  Provisions for safety, security, privacy, and continuity of operations
  in emergencies
\end{enumerate}

\subsection{3.4 Users or involved personnel.}

This paragraph shall describe the types of users of the system, or
personnel involved in the current situation, including, as applicable,
organizational structures, training/skills, responsibilities,
activities, and interactions with one another.

\subsection{3.5 Support concept.}

This paragraph shall provide an overview of the support concept for the
current system, including, as applicable to this document, support
agency(ies); facilities; equipment; support software; repair/replacement
criteria; maintenance levels and cycles; and storage, distribution, and
supply methods.

\section{4. Justification for and nature of changes.}

This section shall be divided into the following paragraphs.

\subsection{4.1 Justification for change.}

This paragraph shall:

\begin{enumerate}
\itemsep1pt\parskip0pt\parsep0pt
\item
  Describe new or modified aspects of user needs, threats, missions,
  objectives, environ-ments, interfaces, personnel or other factors that
  require a new or modified system
\item
  Summarize deficiencies or limitations in the current system or
  situation that make it unable to respond to these factors
\end{enumerate}

\subsection{4.2 Description of needed changes.}

This paragraph shall summarize new or modified capabilities/functions,
processes, interfaces, or other changes needed to respond to the factors
identified in 4.1.

\subsection{4.3 Priorities among the changes.}

This paragraph shall identify priorities among the needed changes. It
shall, for example, identify each change as essential, desirable, or
optional, and prioritize the desirable and optional changes.

\subsection{4.4 Changes considered but not included.}

This paragraph shall identify changes considered but not included in
4.2, and rationale for not including them.

\subsection{4.5 Assumptions and constraints.}

This paragraph shall identify any assumptions and constraints applicable
to the changes identi-fied in this section.

\section{5. Concept for a new or modified system.}

This section shall be divided into the following paragraphs to describe
a new or modified system.

\subsection{5.1 Background, objectives, and scope.}

This paragraph shall describe the background, mission or objectives, and
scope of the new or modified system.

\subsection{5.2 Operational policies and constraints.}

This paragraph shall describe any operational policies and constraints
that apply to the new or modified system.

\subsection{5.3 Description of the new or modified system.}

This paragraph shall provide a description of the new or modified
system, identifying differences associated with different states or
modes of operation (for example, regular, maintenance, train-ing,
degraded, emergency, alternative-site, wartime, peacetime). The
distinction between states and modes is arbitrary. A system may be
described in terms of states only, modes only, states within modes,
modes within states, or any other scheme that is useful. If the system
operates without states or modes, this paragraph shall so state, without
the need to create artificial distinctions. The description shall
include, as applicable:

\begin{enumerate}
\itemsep1pt\parskip0pt\parsep0pt
\item
  The operational environment and its characteristics
\item
  Major system components and the interconnections among these
  components
\item
  Interfaces to external systems or procedures
\item
  Capabilities/functions of the new or modified system
\item
  Charts and accompanying descriptions depicting inputs, outputs, data
  flow, and manual and automated processes sufficient to understand the
  new or modified system or situation from the user's point of view
\item
  Performance characteristics, such as speed, throughput, volume,
  frequency
\item
  Quality attributes, such as reliability, maintainability,
  availability, flexibility, portability, usability, efficiency
\item
  Provisions for safety, security, privacy, and continuity of operations
  in emergencies
\end{enumerate}

\subsection{5.4 Users/affected personnel.}

This paragraph shall describe the types of users of the new or modified
system, including, as applicable, organizational structures,
training/skills, responsibilities, and interactions with one another.

\subsection{5.5 Support concept.}

This paragraph shall provide an overview of the support concept for the
new or modified system, including, as applicable, support agency(ies);
facilities; equipment; support software; repair/replacement criteria;
maintenance levels and cycles; and storage, distribution, and supply
methods.

\section{6. Operational scenarios.}

This section shall describe one or more operational scenarios that
illustrate the role of the new or modified system, its interaction with
users, its interface to other systems, and all states or modes
identified for the system. The scenarios shall include events, actions,
stimuli, information, interactions, etc., as applicable. Reference may
be made to other media, such as videos, to provide part or all of this
information.

\section{7. Summary of impacts.}

This section shall be divided into the following paragraphs.

\subsection{7.1 Operational impacts.}

This paragraph shall describe anticipated operational impacts on the
user, acquirer, developer, and support agency(ies). These impacts may
include changes in interfaces with computer operating centers; change in
procedures; use of new data sources; changes in quantity, type, and
timing of data to be input to the system; changes in data retention
requirements; and new modes of operation based on peacetime, alert,
wartime, or emergency conditions.

\subsection{7.2 Organizational impacts.}

This paragraph shall describe anticipated organizational impacts on the
user, acquirer, developer, and support agency(ies). These impacts may
include modification of responsibilities; addition or elimination of
responsibilities or positions; need for training or retraining; and
changes in number, skill levels, position identifiers, or location of
personnel in various modes of operation.

\subsection{7.3 Impacts during development.}

This paragraph shall describe anticipated impacts on the user, acquirer,
developer, and support agency(ies) during the development effort. These
impacts may include meetings/discussions regarding the new system;
development or modification of databases; training; parallel operation
of the new and existing systems; impacts during testing of the new
system; and other activities needed to aid or monitor development.

\section{8. Analysis of the proposed system.}

\subsection{8.1 Summary of advantages.}

This paragraph shall provide a qualitative and quantitative summary of
the advantages to be obtained from the new or modified system. This
summary shall include new capabilities, enhanced capabilities, and
improved performance, as applicable, and their relationship to
deficiencies identified in 4.1.

\subsection{8.2 Summary of disadvantages/limitations.}

This paragraph shall provide a qualitative and quantitative summary of
disadvantages or limitations of the new or modified system. These
disadvantages and limitations shall include, as applicable, degraded or
missing capabilities, degraded or less-than-desired performance,
greater-than-desired use of computer hardware resources, undesirable
operational impacts, conflicts with user assumptions, and other
constraints.

\subsection{8.3 Alternatives and trade-offs considered.}

This paragraph shall identify and describe major alternatives considered
to the system or its characteristics, the trade-offs among them, and
rationale for the decisions reached.

\section{9. Notes.}

This section shall contain any general information that aids in
understanding this document (e.g., background information, glossary,
rationale). This section shall include an alphabetical listing of all
acronyms, abbreviations, and their meanings as used in this document and
a list of any terms and definitions needed to understand this document.

\section{A. Appendixes.}

Appendixes may be used to provide information published separately for
convenience in document maintenance (e.g., charts, classified data). As
applicable, each appendix shall be referenced in the main body of the
document where the data would normally have been provided. Appendixes
may be bound as separate documents for ease in handling. Appendixes
shall be lettered alphabetically (A, B, etc.).
