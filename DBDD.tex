\section{1. Scope.}

This section shall be divided into the following paragraphs.

\subsection{1.1 Identification.}

This paragraph shall contain a full identification of the database to
which this document applies, including, as applicable, identification
number(s), title(s), abbreviation(s), version number(s), and release
number(s).

\subsection{1.2 Database overview.}

This paragraph shall briefly state the purpose of the database to which
this document applies. It shall describe the general nature of the
database; summarize the history of its development, use, and
maintenance; identify the project sponsor, acquirer, user, developer,
and support agencies; identify current and planned operating sites; and
list other relevant documents.

\subsection{1.3 Document overview.}

This paragraph shall summarize the purpose and contents of this document
and shall describe any security or privacy considerations associated
with its use.

\section{2. Referenced documents.}

This section shall list the number, title, revision, and date of all
documents referenced in this manual. This section shall also identify
the source for all documents not available through normal Government
stocking activities.

\section{3. Database-wide design decisions.}

This section shall be divided into paragraphs as needed to present
database-wide design decisions, that is, decisions about the database's
behavioral design (how it will behave, from a user's point of view, in
meeting its requirements, ignoring internal implementation) and other
decisions affecting further design of the database. If all such
decisions are explicit in the system or CSCI requirements, this section
shall so state. Design decisions that respond to requirements designated
critical, such as those for safety, security, or privacy, shall be
placed in separate subparagraphs. If a design decision depends upon
system states or modes, this dependency shall be indicated. If some or
all of the design decisions are described in the documentation of a
custom or commercial database management system (DBMS), they may be
referenced from this section. Design conventions needed to understand
the design shall be presented or referenced. Examples of database-wide
design decisions are the following:

\begin{enumerate}
\itemsep1pt\parskip0pt\parsep0pt
\item
  Design decisions regarding queries or other inputs the database will
  accept and outputs (displays, reports, messages, responses, etc.) it
  will produce, including interfaces with other systems, HWCIs, CSCIs,
  and users (5.x.d of this DID identifies topics to be considered in
  this description). If part or all of this information is given in
  Interface Design Descriptions (IDDs), they may be referenced.
\item
  Design decisions on database behavior in response to each input or
  query, including actions, response times and other performance
  characteristics, selected equations/algorithms/rules, disposition, and
  handling of unallowed inputs
\item
  Design decisions on how databases/data files will appear to the user
  (4.x of this DID identifies topics to be considered in this
  description)
\item
  Design decisions on the database management system to be used
  (including name, version/release) and the type of flexibility to be
  built into the database for adapting to changing requirements
\item
  Design decisions on the levels and types of availability, security,
  privacy, and continuity of operations to be offered by the database
\item
  Design decisions on database distribution (such as client/server),
  master database file updates and maintenance, including maintaining
  consistency, establishing/ reestablishing and maintaining
  synchronization, enforcing integrity and business rules
\item
  Design decisions on backup and restoration including data and process
  distribution strategies, permissible actions during backup and
  restoration, and special considerations for new or non-standard
  technologies such as video and sound
\item
  Design decisions on repacking, sorting, indexing, synchronization, and
  consistency including automated disk management and space reclamation
  considerations, optimizing strategies and considerations, storage and
  size considerations, and population of the database and capture of
  legacy data
\end{enumerate}

\section{4. Detailed design of the database.}

This section shall be divided into paragraphs as needed to describe the
detailed design of the database. The number of levels of design and the
names of those levels shall be based on the design methodology used.
Examples of database design levels include conceptual, internal,
logical, and physical. If part or all of the design depends upon system
states or modes, this dependency shall be indicated. Design conventions
needed to understand the design shall be presented or referenced. \\\\
Note: This DID uses the term ``data element assembly'' to mean any
entity, relation, schema, field, table, array, etc., that has structure
(number/order/grouping of data elements) at a given design level (e.g.,
conceptual, internal, logical, physical) and the term ``data element''
to mean any relation, attribute, field, cell, data element, etc. that
does not have structure at that level.

\subsection{4.x (Name of database design level).}

This paragraph shall identify a database design level and shall describe
the data elements and data element assemblies of the database in the
terminology of the selected design method. The information shall include
the following, as applicable, presented in any order suited to the
information to be provided:

\begin{enumerate}
\itemsep1pt\parskip0pt\parsep0pt
\item
  Characteristics of individual data elements in the database design,
  such as:

  \begin{enumerate}
  \itemsep1pt\parskip0pt\parsep0pt
  \item
    Names/identifiers

    \begin{enumerate}
    \itemsep1pt\parskip0pt\parsep0pt
    \item
      Project-unique identifier
    \item
      Non-technical (natural-language) name
    \item
      DoD standard data element name
    \item
      Technical name (e.g., field name in the database)
    \item
      Abbreviation or synonymous names
    \end{enumerate}
  \item
    Data type (alphanumeric, integer, etc.)
  \item
    Size and format (such as length and punctuation of a character
    string)
  \item
    Units of measurement (such as meters, dollars, nanoseconds)
  \item
    Range or enumeration of possible values (such as 0-99)
  \item
    Accuracy (how correct) and precision (number of significant digits)
  \item
    Priority, timing, frequency, volume, sequencing, and other
    constraints, such as whether the data element may be updated and
    whether business rules apply
  \item
    Security and privacy constraints
  \item
    Sources (setting/sending entities) and recipients (using/receiving
    entities)
  \end{enumerate}
\item
  Characteristics of data element assemblies (records, messages, files,
  arrays, displays, reports, etc.) in the database design, such as:

  \begin{enumerate}
  \itemsep1pt\parskip0pt\parsep0pt
  \item
    Names/identifiers

    \begin{enumerate}
    \itemsep1pt\parskip0pt\parsep0pt
    \item
      Project-unique identifier
    \item
      Non-technical (natural language) name
    \item
      Technical name (e.g., record or data structure name in code or
      database)
    \item
      Abbreviations or synonymous names
    \end{enumerate}
  \item
    Data elements in the assembly and their structure (number, order,
    grouping)
  \item
    Medium (such as disk) and structure of data elements/assemblies on
    the medium
  \item
    Visual and auditory characteristics of displays and other outputs
    (such as colors, layouts, fonts, icons and other display elements,
    beeps, lights)
  \item
    Relationships among assemblies, such as sorting/access
    characteristics
  \item
    Priority, timing, frequency, volume, sequencing, and other
    constraints, such as whether the assembly may be updated and whether
    business rules apply
  \item
    Security and privacy constraints
  \item
    Sources (setting/sending entities) and recipients (using/receiving
    entities)
  \end{enumerate}
\end{enumerate}

\section{5. Detailed design of software units used for database access
or manipulation.}

This section shall be divided into the following paragraphs to describe
each software unit used for database access or manipulation. If part or
all of this information is provided elsewhere, such as in a Software
Design Description (SDD), the SDD for a customized DBMS, or the user
manual of a commercial DBMS, that information may be referenced rather
than repeated here. If part or all of the design depends upon system
states or modes, this dependency shall be indicated. If design
information falls into more than one paragraph, it may be presented once
and referenced from the other paragraphs. Design conventions needed to
understand the design shall be presented or referenced.

\subsection{5.x (Project-unique identifier of a software unit, or
designator for a group of software units).}

This paragraph shall identify a software unit by project-unique
identifier and shall describe the unit. The description shall include
the following information, as applicable. Alternatively, this paragraph
may designate a group of software units and identify and describe the
software units in subparagraphs. Software units that contain other
software units may reference the descriptions of those units rather than
repeating information.

\begin{enumerate}
\itemsep1pt\parskip0pt\parsep0pt
\item
  Unit design decisions, if any, such as algorithms to be used, if not
  previously selected
\item
  Any constraints, limitations, or unusual features in the design of the
  software unit
\item
  The programming language to be used and rationale for its use if other
  than the specified CSCI language
\item
  If the software unit consists of or contains procedural commands (such
  as menu selections in a database management system (DBMS) for defining
  forms and reports, on-line DBMS queries for database access and
  manipulation, input to a graphical user interface (GUI) builder for
  automated code generation, commands to the operating system, or shell
  scripts), a list of the procedural commands and a reference to user
  manuals or other documents that explain them
\item
  If the software unit contains, receives, or outputs data, a
  description of its inputs, outputs, and other data elements and data
  element assemblies, as applicable. Data local to the software unit
  shall be described separately from data input to or output from the
  software unit. Interface characteristics may be provided here or by
  referencing Interface Design Description(s). If a given interfacing
  entity is not covered by this DBDD (for example, an external system)
  but its interface characteristics need to be mentioned to describe
  software units that are, these characteristics shall be stated as
  assumptions or as ``When {[}the entity not covered{]} does this,
  {[}the software unit{]} will\ldots{}.'' This paragraph may reference
  other documents (such as data dictionaries, standards for protocols,
  and standards for user interfaces) in place of stating the information
  here. The design description shall include the following, as
  applicable, presented in any order suited to the information to be
  provided, and shall note any differences in these characteristics from
  the point of view of the interfacing entities (such as different
  expectations about the size, frequency, or other characteristics of
  data elements):

  \begin{enumerate}
  \itemsep1pt\parskip0pt\parsep0pt
  \item
    Project-unique identifier for the interface
  \item
    Identification of the interfacing entities (software units,
    configuration items, users, etc.) by name, number, version, and
    documentation references, as applicable
  \item
    Priority assigned to the interface by the interfacing entity(ies)
  \item
    Type of interface (such as real-time data transfer,
    storage-and-retrieval of data, etc.) to be implemented
  \item
    Characteristics of individual data elements that the interfacing
    entity(ies) will provide, store, send, access, receive, etc.
    Paragraph 4.x.a of this DID identifies topics to be covered in this
    description.
  \item
    Characteristics of data element assemblies (records, messages,
    files, arrays, displays, reports, etc.) that the interfacing
    entity(ies) will provide, store, send, access, receive, etc.
    Paragraph 4.x.b of this DID identifies topics to be covered in this
    description.
  \item
    Characteristics of communication methods that the interfacing
    entity(ies) will use for the interface, such as:

    \begin{enumerate}
    \itemsep1pt\parskip0pt\parsep0pt
    \item
      Project-unique identifier(s)
    \item
      Communication links/bands/frequencies/media and their
      characteristics
    \item
      Message formatting
    \item
      Flow control (such as sequence numbering and buffer allocation)
    \item
      Data transfer rate, whether periodic/aperiodic, and interval
      between transfers
    \item
      Routing, addressing, and naming conventions
    \item
      Transmission services, including priority and grade
    \item
      Safety/security/privacy considerations, such as encryption, user
      authentication, compartmentalization, and auditing
    \end{enumerate}
  \item
    Characteristics of protocols that the interfacing entity(ies) will
    use for the interface, such as:

    \begin{enumerate}
    \itemsep1pt\parskip0pt\parsep0pt
    \item
      Project-unique identifier(s)
    \item
      Priority/layer of the protocol
    \item
      Packeting, including fragmentation and reassembly, routing, and
      addressing
    \item
      Legality checks, error control, and recovery procedures
    \item
      Synchronization, including connection establishment, maintenance,
      termination
    \item
      Status, identification, and any other reporting features
    \end{enumerate}
  \item
    Other characteristics, such as physical compatibility of the
    interfacing entity(ies) (dimensions, tolerances, loads, voltages,
    plug compatibility, etc.)
  \end{enumerate}
\item
  If the software unit contains logic, the logic to be used by the
  software unit, including, as applicable:

  \begin{enumerate}
  \itemsep1pt\parskip0pt\parsep0pt
  \item
    Conditions in effect within the software unit when its execution is
    initiated
  \item
    Conditions under which control is passed to other software units
  \item
    Response and response time to each input, including data conversion,
    renaming, and data transfer operations
  \item
    Sequence of operations and dynamically controlled sequencing during
    the software unit's operation, including:

    \begin{enumerate}
    \itemsep1pt\parskip0pt\parsep0pt
    \item
      The method for sequence control
    \item
      The logic and input conditions of that method, such as timing
      variations, priority assignments
    \item
      Data transfer in and out of memory
    \item
      The sensing of discrete input signals, and timing relationships
      between interrupt operations within the software unit
    \end{enumerate}
  \item
    Exception and error handling
  \end{enumerate}
\end{enumerate}

\section{6. Requirements traceability.}

This section shall contain:

\begin{enumerate}
\itemsep1pt\parskip0pt\parsep0pt
\item
  Traceability from each database or other software unit covered by this
  DBDD to the system or CSCI requirements it addresses.
\item
  Traceability from each system or CSCI requirement that has been
  allocated to a database or other software unit covered in this DBDD to
  the database or other software units that address it.
\end{enumerate}

\section{7. Notes.}

This section shall contain any general information that aids in
understanding this document (e.g., background information, glossary,
rationale). This section shall include an alphabetical listing of all
acronyms, abbreviations, and their meanings as used in this document and
a list of any terms and definitions needed to understand this document.

\section{A. Appendixes.}

Appendixes may be used to provide information published separately for
convenience in document maintenance (e.g., charts, classified data). As
applicable, each appendix shall be referenced in the main body of the
document where the data would normally have been provided. Appendixes
may be bound as separate documents for ease in handling. Appendixes
shall be lettered alphabetically (A, B, etc.).
