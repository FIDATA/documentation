\documentclass{fidata-report-template}

\begin{document}

\section{Scope}

This section shall be divided into the following paragraphs.

\subsection{Identification}

This paragraph shall contain a full identification of the system and the
software to which this document applies, including, as applicable,
identification number(s), title(s), abbreviation(s), version number(s),
and release number(s).

\subsection{System overview}

This paragraph shall briefly state the purpose of the system and the
software to which this document applies. It shall describe the general
nature of the system and software; summarize the history of system
development, operation, and maintenance; identify the project sponsor,
acquirer, user, developer, and support agencies; identify current and
planned operating sites; and list other relevant documents.

\subsection{Document overview}

This paragraph shall summarize the purpose and contents of this document
and shall describe any security or privacy considerations associated
with its use.

\subsection{Relationship to other plans}

This paragraph shall describe the relationship, if any, of the SDP to
other project management plans.

\section{Referenced documents}

This section shall list the number, title, revision, and date of all
documents referenced in this plan. This section shall also identify the
source for all documents not available through normal Government
stocking activities.

\section{Overview of required work}

This section shall be divided into paragraphs as needed to establish the
context for the planning described in later sections. It shall include,
as applicable, an overview of:

\begin{enumerate}
\itemsep1pt\parskip0pt\parsep0pt
\item
  Requirements and constraints on the system and software to be
  developed
\item
  Requirements and constraints on project documentation
\item
  Position of the project in the system life cycle
\item
  The selected program/acquisition strategy or any requirements or
  constraints on it
\item
  Requirements and constraints on project schedules and resources
\item
  Other requirements and constraints, such as on project security,
  privacy, methods, standards, interdependencies in hardware and
  software development, etc.
\end{enumerate}

\section{Plans for performing general software development
activities}

This section shall be divided into the following paragraphs. Provisions
corresponding to non-required activities may be satisfied by the words
``Not applicable.'' If different builds or different software on the
project require different planning, these differences shall be noted in
the paragraphs. In addition to the content specified below, each
paragraph shall identify applicable risks/uncertainties and plans for
dealing with them.

\subsection{Software development process}

This paragraph shall describe the software development process to be
used. The planning shall cover all contractual clauses concerning this
topic, identifying planned builds, if applicable, their objectives, and
the software development activities to be performed in each build.

\subsection{General plans for software development}

This paragraph shall be divided into the following subparagraphs.

\subsubsection{Software development methods}

This paragraph shall describe or reference the software development
methods to be used. Included shall be descriptions of the manual and
automated tools and procedures to be used in support of these methods.
The methods shall cover all contractual clauses concerning this topic.
Reference may be made to other paragraphs in this plan if the methods
are better described in context with the activities to which they will
be applied.

\subsubsection{Standards for software products}

This paragraph shall describe or reference the standards to be followed
for representing requirements, design, code, test cases, test
procedures, and test results. The standards shall cover all contractual
clauses concerning this topic. Reference may be made to other paragraphs
in this plan if the standards are better described in context with the
activities to which they will be applied. Standards for code shall be
provided for each programming language to be used. They shall include at
a minimum:

\begin{enumerate}
\itemsep1pt\parskip0pt\parsep0pt
\item
  Standards for format (such as indentation, spacing, capitalization,
  and order of information)
\item
  Standards for header comments (requiring, for example, name/identifier
  of the code; version identification; modification history; purpose;
  requirements and design decisions implemented; notes on the processing
  (such as algorithms used, assumptions, constraints, limitations, and
  side effects); and notes on the data (inputs, outputs, variables, data
  structures, etc.)
\item
  Standards for other comments (such as required number and content
  expectations)
\item
  Naming conventions for variables, parameters, packages, procedures,
  files, etc.
\item
  Restrictions, if any, on the use of programming language constructs or
  features
\item
  Restrictions, if any, on the complexity of code aggregates
\end{enumerate}

\subsubsection{Reusable software products}

This paragraph shall be divided into the following subparagraphs.

\paragraph{Incorporating reusable software products}

This paragraph shall describe the approach to be followed for
identifying, evaluating, and incorporating reusable software products,
including the scope of the search for such products and the criteria to
be used for their evaluation. It shall cover all contractual clauses
concerning this topic. Candidate or selected reusable software products
known at the time this plan is prepared or updated shall be identified
and described, together with benefits, drawbacks, and restrictions, as
applicable, associated with their use.

\paragraph{Developing reusable software products}

This paragraph shall describe the approach to be followed for
identifying, evaluating, and reporting opportunities for developing
reusable software products. It shall cover all contractual clauses
concerning this topic.

\subsubsection{Handling of critical requirements}

This paragraph shall be divided into the following subparagraphs to
describe the approach to be followed for handling requirements
designated critical. The planning in each subparagraph shall cover all
contractual clauses concerning the identified topic.

\paragraph{Safety assurance}

\paragraph{Security assurance}

\paragraph{Privacy assurance}

\paragraph{Assurance of other critical requirements}

\subsubsection{Computer hardware resource utilization}

This paragraph shall describe the approach to be followed for allocating
computer hardware resources and monitoring their utilization. It shall
cover all contractual clauses concerning this topic.

\subsubsection{Recording rationale}

This paragraph shall describe the approach to be followed for recording
rationale that will be useful to the support agency for key decisions
made on the project. It shall interpret the term ``key decisions'' for
the project and state where the rationale are to be recorded. It shall
cover all contractual clauses concerning this topic.

\subsubsection{Access for acquirer review}

This paragraph shall describe the approach to be followed for providing
the acquirer or its authorized representative access to developer and
subcontractor facilities for review of software products and activities.
It shall cover all contractual clauses concerning this topic.

\section{Plans for performing detailed software development
activities}

This section shall be divided into the following paragraphs. Provisions
corresponding to non-required activities may be satisfied by the words
``Not applicable.'' If different builds or different software on the
project require different planning, these differences shall be noted in
the paragraphs. The discussion of each activity shall include the
approach (methods/procedures/tools) to be applied to: 1) the analysis
or other technical tasks involved, 2) the recording of results, and 3)
the preparation of associated deliverables, if applicable. The
discussion shall also identify applicable risks/uncertainties and plans
for dealing with them. Reference may be made to 4.2.1 if applicable
methods are described there.

\subsection{Project planning and oversight}

This paragraph shall be divided into the following subparagraphs to
describe the approach to be followed for project planning and oversight.
The planning in each subparagraph shall cover all contractual clauses
regarding the identified topic.

\subsubsection{Software development planning (covering updates to this
plan)}

\subsubsection{CSCI test planning}

\subsubsection{System test planning}

\subsubsection{Software installation planning}

\subsubsection{Software transition planning}

\subsubsection{Following and updating plans, including the intervals
for management review}

\subsection{Establishing a software development environment}

This paragraph shall be divided into the following subparagraphs to
describe the approach to be followed for establishing, controlling, and
maintaining a software development environment. The planning in each
subparagraph shall cover all contractual clauses regarding the
identified topic.

\subsubsection{Software engineering environment}

\subsubsection{Software test environment}

\subsubsection{Software development library}

\subsubsection{Software development files}

\subsubsection{Non-deliverable software}

\subsection{System requirements analysis}

This paragraph shall be divided into the following subparagraphs to
describe the approach to be followed for participating in system
requirements analysis. The planning in each subparagraph shall cover all
contractual clauses regarding the identified topic.

\subsubsection{Analysis of user input}

\subsubsection{Operational concept}

\subsubsection{System requirements}

\subsection{System design}

This paragraph shall be divided into the following subparagraphs to
describe the approach to be followed for participating in system design.
The planning in each subparagraph shall cover all contractual clauses
regarding the identified topic.

\subsubsection{System-wide design decisions}
\subsubsection{System architectural design}

\subsection{Software requirements analysis}

This paragraph shall describe the approach to be followed for software
requirements analysis. The approach shall cover all contractual clauses
concerning this topic.

\subsection{Software design}

This paragraph shall be divided into the following subparagraphs to
describe the approach to be followed for software design. The planning
in each subparagraph shall cover all contractual clauses regarding the
identified topic.

\subsubsection{CSCI-wide design decisions}

\subsubsection{CSCI architectural design}

\subsubsection{CSCI detailed design}

\subsection{Software implementation and unit testing}

This paragraph shall be divided into the following subparagraphs to
describe the approach to be followed for software implementation and
unit testing. The planning in each subparagraph shall cover all
contractual clauses regarding the identified topic.

\subsubsection{Software implementation}

\subsubsection{Preparing for unit testing}

\subsubsection{Performing unit testing}

\subsubsection{Revision and retesting}

\subsubsection{Analyzing and recording unit test results}

\subsection{Unit integration and testing}

This paragraph shall be divided into the following subparagraphs to
describe the approach to be followed for unit integration and testing.
The planning in each subparagraph shall cover all contractual clauses
regarding the identified topic.

\subsubsection{Preparing for unit integration and testing}

\subsubsection{Performing unit integration and testing}

\subsubsection{Revision and retesting}

\subsubsection{Analyzing and recording unit integration and test
results}

\subsection{CSCI qualification testing}

This paragraph shall be divided into the following subparagraphs to
describe the approach to be followed for CSCI qualification testing. The
planning in each subparagraph shall cover all contractual clauses
regarding the identified topic.

\subsubsection{Independence in CSCI qualification testing}

\subsubsection{Testing on the target computer system}

\subsubsection{Preparing for CSCI qualification testing}

\subsubsection{Dry run of CSCI qualification testing}

\subsubsection{Performing CSCI qualification testing}

\subsubsection{Revision and retesting}

\subsubsection{Analyzing and recording CSCI qualification test results}

\subsection{CSCI/HWCI integration and testing}

This paragraph shall be divided into the following subparagraphs to
describe the approach to be followed for participating in CSCI/HWCI
integration and testing. The planning in each subparagraph shall cover
all contractual clauses regarding the identified topic.

\subsubsection{Preparing for CSCI/HWCI integration and testing}

\subsubsection{Performing CSCI/HWCI integration and testing}

\subsubsection{Revision and retesting}

\subsubsection{Analyzing and recording CSCI/HWCI integration and test
results}

\subsection{System qualification testing}

This paragraph shall be divided into the following subparagraphs to
describe the approach to be followed for participating in system
qualification testing. The planning in each subparagraph shall cover all
contractual clauses regarding the identified topic.

\subsubsection{Independence in system qualification testing}

\subsubsection{Testing on the target computer system}

\subsubsection{Preparing for system qualification testing}

\subsubsection{Dry run of system qualification testing}

\subsubsection{Performing system qualification testing}

\subsubsection{Revision and retesting}

\subsubsection{Analyzing and recording system qualification test
results}

\subsection{Preparing for software use}

This paragraph shall be divided into the following subparagraphs to
describe the approach to be followed for preparing for software use. The
planning in each subparagraph shall cover all contractual clauses
regarding the identified topic.

\subsubsection{Preparing the executable software}

\subsubsection{Preparing version descriptions for user sites}

\subsubsection{Preparing user manuals}

\subsubsection{Installation at user sites}

\subsection{Preparing for software transition}

This paragraph shall be divided into the following subparagraphs to
describe the approach to be followed for preparing for software
transition. The planning in each subparagraph shall cover all
contractual clauses regarding the identified topic.

\subsubsection{Preparing the executable software}

\subsubsection{Preparing source files}

\subsubsection{Preparing version descriptions for the support site}

\subsubsection{Preparing the ``as built'' CSCI design and other
software support information}

\subsubsection{Updating the system design description}

\subsubsection{Preparing support manuals}

\subsubsection{Transition to the designated support site}

\subsection{Software configuration management}

This paragraph shall be divided into the following subparagraphs to
describe the approach to be followed for software configuration
management. The planning in each subparagraph shall cover all
contractual clauses regarding the identified topic.

\subsubsection{5.14.1 Configuration identification}

\subsubsection{5.14.2 Configuration control}

\subsubsection{5.14.3 Configuration status accounting}

\subsubsection{5.14.4 Configuration audits}

\subsubsection{5.14.5 Packaging, storage, handling, and delivery}

\subsection{Software product evaluation}

This paragraph shall be divided into the following subparagraphs to
describe the approach to be followed for software product evaluation.
The planning in each subparagraph shall cover all contractual clauses
regarding the identified topic.

\subsubsection{In-process and final software product evaluations}

\subsubsection{Software product evaluation records, including items to
be recorded}

\subsubsection{Independence in software product evaluation}

\subsection{Software quality assurance}

This paragraph shall be divided into the following subparagraphs to
describe the approach to be followed for software quality assurance. The
planning in each subparagraph shall cover all contractual clauses
regarding the identified topic.

\subsubsection{Software quality assurance evaluations}

\subsubsection{Software quality assurance records, including items to
be recorded}

\subsubsection{Independence in software quality assurance}

\subsection{Corrective action}

This paragraph shall be divided into the following subparagraphs to
describe the approach to be followed for corrective action. The planning
in each subparagraph shall cover all contractual clauses regarding the
identified topic.

\subsubsection{Problem/change reports, including items to
be recorded (candidate items include project name, originator, problem
number, problem name, software element or document affected, origination
date, category and priority, description, analyst assigned to the
problem, date assigned, date completed, analysis time, recommended
solution, impacts, problem status, approval of solution, follow-up
actions, corrector, correction date, version where corrected, correction
time, description of solution implemented)}

\subsubsection{Corrective action system}

\subsection{Joint technical and management reviews}

This paragraph shall be divided into the following subparagraphs to
describe the approach to be followed for joint technical and management
reviews. The planning in each subparagraph shall cover all contractual
clauses regarding the identified topic.

\subsubsection{Joint technical reviews, including a proposed set of
reviews}

\subsubsection{Joint management reviews, including a proposed set of
reviews}

\subsection{Other software development activities}

This paragraph shall be divided into the following subparagraphs to
describe the approach to be followed for other software development
activities. The planning in each subparagraph shall cover all
contractual clauses regarding the identified topic.

\subsubsection{Risk management, including known risks and corresponding
strategies}

\subsubsection{Software management indicators, including indicators to
be used}

\subsubsection{Security and privacy}

\subsubsection{Subcontractor management}

\subsubsection{Interface with software independent verification and
validation (IV\&V) agents}

\subsubsection{Coordination with associate developers}

\subsubsection{Improvement of project processes}

\subsubsection{Other activities not covered elsewhere in the plan}

\section{Schedules and activity network}

This section shall present:

\begin{enumerate}
\itemsep1pt\parskip0pt\parsep0pt
\item
  Schedule(s) identifying the activities in each build and showing
  initiation of each activity, availability of draft and final
  deliverables and other milestones, and completion of each activity
\item
  An activity network, depicting sequential relationships and
  dependencies among activities and identifying those activities that
  impose the greatest time restrictions on the project
\end{enumerate}

\section{Project organization and resources}

This section shall be divided into the following paragraphs to describe
the project organization and resources to be applied in each build.

\subsection{Project organization}

This paragraph shall describe the organizational structure to be used on
the project, including the organizations involved, their relationships
to one another, and the authority and responsibility of each
organization for carrying out required activities.

\subsection{Project resources}

This paragraph shall describe the resources to be applied to the
project. It shall include, as applicable:

\begin{enumerate}
\itemsep1pt\parskip0pt\parsep0pt
\item
  Personnel resources, including:

  \begin{enumerate}
  \itemsep1pt\parskip0pt\parsep0pt
  \item
    The estimated staff-loading for the project (number of personnel
    over time)
  \item
    The breakdown of the staff-loading numbers by responsibility (for
    example, management, software engineering, software testing,
    software configuration management, software product evaluation,
    software quality assurance)
  \item
    A breakdown of the skill levels, geographic locations, and security
    clearances of personnel performing each responsibility
  \end{enumerate}
\item
  Overview of developer facilities to be used, including geographic
  locations in which the work will be performed, facilities to be used,
  and secure areas and other features of the facilities as applicable to
  the contracted effort.
\item
  Acquirer-furnished equipment, software, services, documentation, data,
  and facilities required for the contracted effort. A schedule
  detailing when these items will be needed shall also be included.
\item
  Other required resources, including a plan for obtaining the
  resources, dates needed, and availability of each resource item.
\end{enumerate}

\section{Notes}

This section shall contain any general information that aids in
understanding this document (e.g., background information, glossary,
rationale). This section shall include an alphabetical listing of all
acronyms, abbreviations, and their meanings as used in this document and
a list of any terms and definitions needed to understand this document.

\appendix

\section{Appendixes}

Appendixes may be used to provide information published separately for
convenience in document maintenance (e.g., charts, classified data). As
applicable, each appendix shall be referenced in the main body of the
document where the data would normally have been provided. Appendixes
may be bound as separate documents for ease in handling. Appendixes
shall be lettered alphabetically (A, B, etc.).

\end{document}
