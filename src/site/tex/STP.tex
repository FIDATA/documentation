\documentclass{fidata-report-template}

\begin{document}

\frontmatter

\title{Software Test Plan}

\date{2014-02-24}

\author{Basil Peace}

\maketitle
\tableofcontents
% \listoffiguresandtables

\section{Scope}

This section shall be divided into the following paragraphs.

\subsection{Identification}

This paragraph shall contain a full identification of the system and the
software to which this document applies, including, as applicable,
identification number(s), title(s), abbreviation(s), version number(s),
and release number(s).

\subsection{System overview}

This paragraph shall briefly state the purpose of the system and the
software to which this document applies. It shall describe the general
nature of the system and software; summarize the history of system
development, operation, and maintenance; identify the project sponsor,
acquirer, user, developer, and support agencies; identify current and
planned operating sites; and list other relevant documents.

\subsection{Document overview}

This paragraph shall summarize the purpose and contents of this document
and shall describe any security or privacy considerations associated
with its use.

\subsection{Relationship to other plans}

This paragraph shall describe the relationship, if any, of the STP to
related project management plans.

\section{Referenced documents}

This section shall list the number, title, revision, and date of all
documents referenced in this plan. This section shall also identify the
source for all documents not available through normal Government
stocking activities.

\section{Software test environment}

This section shall be divided into the following paragraphs to describe
the software test environment at each intended test site. Reference may
be made to the Software Development Plan (SDP) for resources that are
described there.

\subsection{(Name of test site(s))}

This paragraph shall identify one or more test sites to be used for the
testing, and shall be divided into the following subparagraphs to
describe the software test environment at the site(s). If all tests will
be conducted at a single site, this paragraph and its subparagraphs
shall be presented only once. If multiple test sites use the same or
similar software test environments, they may be discussed together.
Duplicative information among test site descriptions may be reduced by
referencing earlier descriptions.

\subsubsection{Software items}

This paragraph shall identify by name, number, and version, as
applicable, the software items (e.g., operating systems, compilers,
communications software, related applications software, databases, input
files, code auditors, dynamic path analyzers, test drivers,
preprocessors, test data generators, test control software, other
special test software, post processors) necessary to perform the planned
testing activities at the test site(s). This paragraph shall describe
the purpose of each item, describe its media (tape, disk, etc.),
identify those that are expected to be supplied by the site, and
identify any classified processing or other security or privacy issues
associated with the software items.

\subsubsection{Hardware and firmware items}

This paragraph shall identify by name, number, and version, as
applicable, the computer hardware, interfacing equipment, communications
equipment, test data reduction equipment, apparatus such as extra
peripherals (tape drives, printers, plotters), test message generators,
test timing devices, test event records, etc., and firmware items that
will be used in the software test environment at the test site(s). This
paragraph shall describe the purpose of each item, state the period of
usage and the number of each item needed, identify those that are
expected to be supplied by the site, and identify any classified
processing or other security or privacy issues associated with the
items.

\subsubsection{Other materials}

This paragraph shall identify and describe any other materials needed
for the testing at the test site(s). These materials may include
manuals, software listings, media containing the software to be tested,
media containing data to be used in the tests, sample listings of
outputs, and other forms or instructions. This paragraph shall identify
those items that are to be delivered to the site and those that are
expected to be supplied by the site. The description shall include the
type, layout, and quantity of the materials, as applicable. This
paragraph shall identify any classified processing or other security or
privacy issues associated with the items.

\subsubsection{Proprietary nature, acquirer's rights, and
licensing}

This paragraph shall identify the proprietary nature, acquirer's rights,
and licensing issues associated with each element of the software test
environment.

\subsubsection{Installation, testing, and control}

This paragraph shall identify the developer's plans for performing each
of the following, possibly in conjunction with personnel at the test
site(s):

\begin{enumerate}
\itemsep1pt\parskip0pt\parsep0pt
\item
  Acquiring or developing each element of the software test environment
\item
  Installing and testing each item of the software test environment
  prior to its use
\item
  Controlling and maintaining each item of the software test environment
\end{enumerate}

\subsubsection{Participating organizations}

This paragraph shall identify the organizations that will participate in
the testing at the test sites(s) and the roles and responsibilities of
each.

\subsubsection{Personnel}

This paragraph shall identify the number, type, and skill level of
personnel needed during the test period at the test site(s), the dates
and times they will be needed, and any special needs, such as multishift
operation and retention of key skills to ensure continuity and
consistency in extensive test programs.

\subsubsection{Orientation plan}

This paragraph shall describe any orientation and training to be given
before and during the testing. This information shall be related to the
personnel needs given in 3.x.7. This training may include user
instruction, operator instruction, maintenance and control group
instruction, and orientation briefings to staff personnel. If extensive
training is anticipated, a separate plan may be developed and referenced
here.

\subsubsection{Tests to be performed}

This paragraph shall identify, by referencing section 4, the tests to be
performed at the test site(s).

\section{Test identification}

This section shall be divided into the following paragraphs to identify
and describe each test to which this STP applies.

\subsection{General information}

This paragraph shall be divided into subparagraphs to present general
information applicable to the overall testing to be performed.

\subsubsection{Test levels}

This paragraph shall describe the levels at which testing will be
performed, for example, CSCI level or system level.

\subsubsection{Test classes}

This paragraph shall describe the types or classes of tests that will be
performed (for example, timing tests, erroneous input tests, maximum
capacity tests).

\subsubsection{General test conditions}

This paragraph shall describe conditions that apply to all of the tests
or to a group of tests. For example: ``Each test shall include nominal,
maximum, and minimum values;'' ``each test of type x shall use live
data;'' ``execution size and time shall be measured for each CSCI.''
Included shall be a statement of the extent of testing to be performed
and rationale for the extent selected. The extent of testing shall be
expressed as a percentage of some well defined total quantity, such as
the number of samples of discrete operating conditions or values, or
other sampling approach. Also included shall be the approach to be
followed for retesting/regression testing.

\subsubsection{Test progression}

In cases of progressive or cumulative tests, this paragraph shall
explain the planned sequence or progression of tests.

\subsubsection{Data recording, reduction, and analysis}

This paragraph shall identify and describe the data recording,
reduction, and analysis procedures to be used during and after the tests
identified in this STP. These procedures shall include, as applicable,
manual, automatic, and semi-automatic techniques for recording test
results, manipulating the raw results into a form suitable for
evaluation, and retaining the results of data reduction and analysis.

\subsection{Planned tests}

This paragraph shall be divided into the following subparagraphs to
describe the total scope of the planned testing.

\subsubsection{(Item(s) to be tested)}

This paragraph shall identify a CSCI, subsystem, system, or other entity
by name and project unique identifier, and shall be divided into the
following subparagraphs to describe the testing planned for the item(s).
(Note: the ``tests'' in this plan are collections of test cases. There
is no intent to describe each test case in this document.)

\paragraph{(Project-unique identifier of a test)}

This paragraph shall identify a test by project unique identifier and
shall provide the information specified below for the test. Reference
may be made as needed to the general information in 4.1.

\begin{enumerate}
\itemsep1pt\parskip0pt\parsep0pt
\item
  Test objective
\item
  Test level
\item
  Test type or class
\item
  Qualification method(s) as specified in the requirements specification
\item
  Identifier of the CSCI requirements and, if applicable, software
  system requirements addressed by this test. (Alternatively, this
  information may be provided in Section 6.)
\item
  Special requirements (for example, 48 hours of continuous facility
  time, weapon simulation, extent of test, use of a special input or
  database)
\item
  Type of data to be recorded
\item
  Type of data recording/reduction/analysis to be employed
\item
  Assumptions and constraints, such as anticipated limitations on the
  test due to system or test conditions--timing, interfaces, equipment,
  personnel, database, etc.
\item
  Safety, security, and privacy considerations associated with the test
\end{enumerate}

\section{Test schedules}

This section shall contain or reference the schedules for conducting the
tests identified in this plan. It shall include:

\begin{enumerate}
\itemsep1pt\parskip0pt\parsep0pt
\item
  A listing or chart depicting the sites at which the testing will be
  scheduled and the time frames during which the testing will be
  conducted
\item
  A schedule for each test site depicting the activities and events
  listed below, as applicable, in chronological order with supporting
  narrative as necessary:

  \begin{enumerate}
  \itemsep1pt\parskip0pt\parsep0pt
  \item
    On site test period and periods assigned to major portions of the
    testing
  \item
    Pretest on site period needed for setting up the software test
    environment and other equipment, system debugging, orientation, and
    familiarization
  \item
    Collection of database/data file values, input values, and other
    operational data needed for the testing
  \item
    Conducting the tests, including planned retesting
  \item
    Preparation, review, and approval of the Software Test Report (STR)
  \end{enumerate}
\end{enumerate}

\section{Requirements traceability}

This paragraph shall contain:

\begin{enumerate}
\itemsep1pt\parskip0pt\parsep0pt
\item
  Traceability from each test identified in this plan to the CSCI
  requirements and, if applicable, software system requirements it
  addresses. (Alternatively, this traceability may be provided in
  4.2.x.y and referenced from this paragraph.)
\item
  Traceability from each CSCI requirement and, if applicable, each
  software system requirement covered by this test plan to the test(s)
  that address it. The traceability shall cover the CSCI requirements in
  all applicable Software Requirements Specifications (SRSs) and
  associated Interface Requirements Specifications (IRSs), and, for
  software systems, the system requirements in all applicable System/
  Subsystem Specifications (SSSs) and associated system-level IRSs.
\end{enumerate}

\section{Notes}

This section shall contain any general information that aids in
understanding this document (e.g., background information, glossary,
rationale). This section shall include an alphabetical listing of all
acronyms, abbreviations, and their meanings as used in this document and
a list of any terms and definitions needed to understand this document.

\appendix

\section{Appendixes}

Appendixes may be used to provide information published separately for
convenience in document maintenance (e.g., charts, classified data). As
applicable, each appendix shall be referenced in the main body of the
document where the data would normally have been provided. Appendixes
may be bound as separate documents for ease in handling. Appendixes
shall be lettered alphabetically (A, B, etc.).

\end{document}
