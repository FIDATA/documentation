\documentclass{fidata-report-template}

\begin{document}

\frontmatter

\title{Software Test Description}

\date{2014-02-24}

\author{Basil Peace}

\maketitle
\tableofcontents
% \listoffiguresandtables

\section{Scope}

This section shall be divided into the following paragraphs.

\subsection{Identification}

This paragraph shall contain a full identification of the system and the
software to which this document applies, including, as applicable,
identification number(s), title(s), abbreviation(s), version number(s),
and release number(s).

\subsection{System overview}

This paragraph shall briefly state the purpose of the system and the
software to which this document applies. It shall describe the general
nature of the system and software; summarize the history of system
development, operation, and maintenance; identify the project sponsor,
acquirer, user, developer, and support agencies; identify current and
planned operating sites; and list other relevant documents.

\subsection{Document overview}

This paragraph shall summarize the purpose and contents of this document
and shall describe any security or privacy considerations associated
with its use.

\section{Referenced documents}

This section shall list the number, title, revision, and date of all
documents referenced in this document. This section shall also identify
the source for all documents not available through normal Government
stocking activities.

\section{Test preparations}

This section shall be divided into the following paragraphs. Safety
precautions, marked by WARNING or CAUTION, and security and privacy
considerations shall be included as applicable.

\subsection{(Project unique identifier of a test)}

This paragraph shall identify a test by project-unique identifier, shall
provide a brief description, and shall be divided into the following
subparagraphs. When the information required duplicates information
previously specified for another test, that information may be
referenced rather than repeated.

\subsubsection{Hardware preparation}

This paragraph shall describe the procedures necessary to prepare the
hardware for the test. Reference may be made to published operating
manuals for these procedures. The following shall be provided, as
applicable:

\begin{enumerate}
\itemsep1pt\parskip0pt\parsep0pt
\item
  The specific hardware to be used, identified by name and, if
  applicable, number
\item
  Any switch settings and cabling necessary to connect the hardware
\item
  One or more diagrams to show hardware, interconnecting control, and
  data paths
\item
  Step-by-step instructions for placing the hardware in a state of
  readiness
\end{enumerate}

\subsubsection{Software preparation}

This paragraph shall describe the procedures necessary to prepare the
item(s) under test and any related software, including data, for the
test. Reference may be made to published software manuals for these
procedures. The following information shall be provided, as applicable:

\begin{enumerate}
\itemsep1pt\parskip0pt\parsep0pt
\item
  The specific software to be used in the test
\item
  The storage medium of the item(s) under test (e.g., magnetic tape,
  diskette)
\item
  The storage medium of any related software (e.g., simulators, test
  drivers, databases)
\item
  Instructions for loading the software, including required sequence
\item
  Instructions for software initialization common to more than one test
  case
\end{enumerate}

\subsubsection{Other pre test preparations}

This paragraph shall describe any other pre-test personnel actions,
preparations, or procedures necessary to perform the test.

\section{Test descriptions}

This section shall be divided into the following paragraphs. Safety
precautions, marked by WARNING or CAUTION, and security and privacy
considerations shall be included as applicable.

\subsection{(Project unique identifier of a test)}

This paragraph shall identify a test by project unique identifier and
shall be divided into the following subparagraphs. When the required
information duplicates information previously provided, that information
may be referenced rather than repeated.

\subsubsection{(Project-unique identifier of a test case)}

This paragraph shall identify a test case by project unique identifier,
state its purpose, and provide a brief description. The following
subparagraphs shall provide a detailed description of the test case.

\paragraph{Requirements addressed}

This paragraph shall identify the CSCI or system requirements addressed
by the test case. (Alternatively, this information may be provided in
5.a.)

\paragraph{Prerequisite conditions}

This paragraph shall identify any prerequisite conditions that must be
established prior to performing the test case. The following
considerations shall be discussed, as applicable:

\begin{enumerate}
\itemsep1pt\parskip0pt\parsep0pt
\item
  Hardware and software configuration
\item
  Flags, initial breakpoints, pointers, control parameters, or initial
  data to be set/reset prior to test commencement
\item
  Preset hardware conditions or electrical states necessary to run the
  test case
\item
  Initial conditions to be used in making timing measurements
\item
  Conditioning of the simulated environment
\item
  Other special conditions peculiar to the test case
\end{enumerate}

\paragraph{Test inputs}

This paragraph shall describe the test inputs necessary for the test
case. The following shall be provided, as applicable:

\begin{enumerate}
\itemsep1pt\parskip0pt\parsep0pt
\item
  Name, purpose, and description (e.g., range of values, accuracy) of
  each test input
\item
  Source of the test input and the method to be used for selecting the
  test input
\item
  Whether the test input is real or simulated
\item
  Time or event sequence of test input
\item
  The manner in which the input data will be controlled to:

  \begin{enumerate}
  \itemsep1pt\parskip0pt\parsep0pt
  \item
    Test the item(s) with a minimum/reasonable number of data types and
    values
  \item
    Exercise the item(s) with a range of valid data types and values
    that test for overload, saturation, and other ``worst case'' effects
  \item
    Exercise the item(s) with invalid data types and values to test for
    appropriate handling of irregular inputs
  \item
    Permit retesting, if necessary
  \end{enumerate}
\end{enumerate}

\paragraph{Expected test results}

This paragraph shall identify all expected test results for the test
case. Both intermediate and final test results shall be provided, as
applicable.

\paragraph{Criteria for evaluating results}

This paragraph shall identify the criteria to be used for evaluating the
intermediate and final results of the test case. For each test result,
the following information shall be provided, as applicable:

\begin{enumerate}
\itemsep1pt\parskip0pt\parsep0pt
\item
  The range or accuracy over which an output can vary and still be
  acceptable
\item
  Minimum number of combinations or alternatives of input and output
  conditions that constitute an acceptable test result
\item
  Maximum/minimum allowable test duration, in terms of time or number of
  events
\item
  Maximum number of interrupts, halts, or other system breaks that may
  occur
\item
  Allowable severity of processing errors
\item
  Conditions under which the result is inconclusive and re testing is to
  be performed
\item
  Conditions under which the outputs are to be interpreted as indicating
  irregularities in input test data, in the test database/data files, or
  in test procedures
\item
  Allowable indications of the control, status, and results of the test
  and the readiness for the next test case (may be output of auxiliary
  test software)
\item
  Additional criteria not mentioned above.
\end{enumerate}

\paragraph{Test procedure}

This paragraph shall define the test procedure for the test case. The
test procedure shall be defined as a series of individually numbered
steps listed sequentially in the order in which the steps are to be
performed. For convenience in document maintenance, the test procedures
may be included as an appendix and referenced in this paragraph. The
appropriate level of detail in each test procedure depends on the type
of software being tested. For some software, each keystroke may be a
separate test procedure step; for most software, each step may include a
logically related series of keystrokes or other actions. The appropriate
level of detail is the level at which it is useful to specify expected
results and compare them to actual results. The following shall be
provided for each test procedure, as applicable:

\begin{enumerate}
\itemsep1pt\parskip0pt\parsep0pt
\item
  Test operator actions and equipment operation required for each step,
  including commands, as applicable, to:

  \begin{enumerate}
  \itemsep1pt\parskip0pt\parsep0pt
  \item
    Initiate the test case and apply test inputs
  \item
    Inspect test conditions
  \item
    Perform interim evaluations of test results
  \item
    Record data
  \item
    Halt or interrupt the test case
  \item
    Request data dumps or other aids, if needed
  \item
    Modify the database/data files
  \item
    Repeat the test case if unsuccessful
  \item
    Apply alternate modes as required by the test case
  \item
    Terminate the test case
  \end{enumerate}
\item
  Expected result and evaluation criteria for each step
\item
  If the test case addresses multiple requirements, identification of
  which test procedure step(s) address which requirements.
  (Alternatively, this information may be provided in 5.)
\item
  Actions to follow in the event of a program stop or indicated error,
  such as:

  \begin{enumerate}
  \itemsep1pt\parskip0pt\parsep0pt
  \item
    Recording of critical data from indicators for reference purposes
  \item
    Halting or pausing time sensitive test support software and test
    apparatus
  \item
    Collection of system and operator records of test results
  \end{enumerate}
\item
  Procedures to be used to reduce and analyze test results to accomplish
  the following, as applicable:

  \begin{enumerate}
  \itemsep1pt\parskip0pt\parsep0pt
  \item
    Detect whether an output has been produced
  \item
    Identify media and location of data produced by the test case
  \item
    Evaluate output as a basis for continuation of test sequence
  \item
    Evaluate test output against required output
  \end{enumerate}
\end{enumerate}

\paragraph{Assumptions and constraints}

This paragraph shall identify any assumptions made and constraints or
limitations imposed in the description of the test case due to system or
test conditions, such as limitations on timing, interfaces, equipment,
personnel, and database/data files. If waivers or exceptions to
specified limits and parameters are approved, they shall be identified
and this paragraph shall address their effects and impacts upon the test
case.

\section{Requirements traceability}

This paragraph shall contain:

\begin{enumerate}
\itemsep1pt\parskip0pt\parsep0pt
\item
  Traceability from each test case in this STD to the system or CSCI
  requirements it addresses. If a test case addresses multiple
  requirements, traceability from each set of test procedure steps to
  the requirement(s) addressed. (Alternatively, this traceability may be
  provided in 4.x.y.1.)
\item
  Traceability from each system or CSCI requirement covered by this STD
  to the test case(s) that address it. For CSCI testing, traceability
  from each CSCI requirement in the CSCI's Software Requirements
  Specification (SRS) and associated Interface Requirements
  Specifications (IRSs). For system testing, traceability from each
  system requirement in the system's System/Subsystem Specification
  (SSS) and associated IRSs. If a test case addresses multiple
  requirements, the traceability shall indicate the particular test
  procedure steps that address each requirement.
\end{enumerate}

\section{Notes}

This section shall contain any general information that aids in
understanding this document (e.g., background information, glossary,
rationale). This section shall include an alphabetical listing of all
acronyms, abbreviations, and their meanings as used in this document and
a list of any terms and definitions needed to understand this document.

\appendix

\section{Appendixes}

Appendixes may be used to provide information published separately for
convenience in document maintenance (e.g., charts, classified data). As
applicable, each appendix shall be referenced in the main body of the
document where the data would normally have been provided. Appendixes
may be bound as separate documents for ease in handling. Appendixes
shall be lettered alphabetically (A, B, etc.).

\end{document}
