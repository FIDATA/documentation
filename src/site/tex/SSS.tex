\documentclass{fidata-report-template}

\begin{document}

\section{Scope}

This section shall be divided into the following paragraphs.

\subsection{Identification}

This paragraph shall contain a full identification of the system to
which this document applies, including, as applicable, identification
number(s), title(s), abbreviation(s), version number(s), and release
number(s).

\subsection{System overview}

This paragraph shall briefly state the purpose of the system to which
this document applies. It shall describe the general nature of the
system; summarize the history of system development, operation, and
maintenance; identify the project sponsor, acquirer, user, developer,
and support agencies; identify current and planned operating sites; and
list other relevant documents.

\subsection{Document overview}

This paragraph shall summarize the purpose and contents of this document
and shall describe any security or privacy considerations associated
with its use.

\section{Referenced documents}

This section shall list the number, title, revision, and date of all
documents referenced in this specification. This section shall also
identify the source for all documents not available through normal
Government stocking activities.

\section{Requirements}

This section shall be divided into the following paragraphs to specify
the system requirements, that is, those characteristics of the system
that are conditions for its acceptance. Each requirement shall be
assigned a project-unique identifier to support testing and traceability
and shall be stated in such a way that an objective test can be defined
for it. Each requirement shall be annotated with associated
qualification method(s) (see section 4) and, for subsystems,
traceability to system requirements (see section 5.a), if not provided
in those sections. The degree of detail to be provided shall be guided
by the following rule: Include those characteristics of the system that
are conditions for system acceptance; defer to design descriptions those
characteristics that the acquirer is willing to leave up to the
developer. If there are no requirements in a given paragraph, the
paragraph shall so state. If a given requirement fits into more than one
paragraph, it may be stated once and referenced from the other
paragraphs.

\subsection{Required states and modes}

If the system is required to operate in more than one state or mode
having requirements distinct from other states or modes, this paragraph
shall identify and define each state and mode. Examples of states and
modes include: idle, ready, active, post-use analysis, training,
degraded, emergency, backup, wartime, peacetime. The distinction between
states and modes is arbitrary. A system may be described in terms of
states only, modes only, states within modes, modes within states, or
any other scheme that is useful. If no states or modes are required,
this paragraph shall so state, without the need to create artificial
distinctions. If states and/or modes are required, each requirement or
group of requirements in this specification shall be correlated to the
states and modes. The correlation may be indicated by a table or other
method in this paragraph, in an appendix referenced from this paragraph,
or by annotation of the requirements in the paragraphs where they
appear.

\subsection{System capability requirements}

This paragraph shall be divided into subparagraphs to itemize the
requirements associated with each capability of the system. A
``capability'' is defined as a group of related requirements. The word
``capability'' may be replaced with ``function,'' ``subject,''
``object,'' or other term useful for presenting the requirements.

\subsubsection{(System capability)}

This paragraph shall identify a required system capability and shall
itemize the requirements associated with the capability. If the
capability can be more clearly specified by dividing it into constituent
capabilities, the constituent capabilities shall be specified in
subparagraphs. The requirements shall specify required behavior of the
system and shall include applicable parameters, such as response times,
throughput times, other timing constraints, sequencing, accuracy,
capacities (how much/how many), priorities, continuous operation
requirements, and allowable deviations based on operating conditions.
The requirements shall include, as applicable, required behavior under
unexpected, unallowed, or ``out of bounds'' conditions, requirements for
error handling, and any provisions to be incorporated into the system to
provide continuity of operations in the event of emergencies. Paragraph
3.3.x of this DID provides a list of topics to be considered when
specifying requirements regarding inputs the system must accept and
outputs it must produce.

\subsection{System external interface requirements}

This paragraph shall be divided into subparagraphs to specify the
requirements, if any, for the system's external interfaces. This
paragraph may reference one or more Interface Requirements
Specifications (IRSs) or other documents containing these requirements.

\subsubsection{Interface identification and diagrams}

This paragraph shall identify the required external interfaces of the
system. The identification of each interface shall include a
project-unique identifier and shall designate the interfacing entities
(systems, configuration items, users, etc.) by name, number, version,
and documentation references, as applicable. The identification shall
state which entities have fixed interface characteristics (and therefore
impose interface requirements on interfacing entities) and which are
being developed or modified (thus having interface requirements imposed
on them). One or more interface diagrams shall be provided to depict the
interfaces.

\subsubsection{(Project unique identifier of interface)}

This paragraph (beginning with 3.3.2) shall identify a system external
interface by project unique identifier, shall briefly identify the
interfacing entities, and shall be divided into subparagraphs as needed
to state the requirements imposed on the system to achieve the
interface. Interface characteristics of the other entities involved in
the interface shall be stated as assumptions or as ``When {[}the entity
not covered{]} does this, the system shall\ldots{},'' not as
requirements on the other entities. This paragraph may reference other
documents (such as data dictionaries, standards for communication
protocols, and standards for user interfaces) in place of stating the
information here. The requirements shall include the following, as
applicable, presented in any order suited to the requirements, and shall
note any differences in these characteristics from the point of view of
the interfacing entities (such as different expectations about the size,
frequency, or other characteristics of data elements):

\begin{enumerate}
\itemsep1pt\parskip0pt\parsep0pt
\item
  Priority that the system must assign the interface
\item
  Requirements on the type of interface (such as real-time data
  transfer, storage-and-retrieval of data, etc.) to be implemented
\item
  Required characteristics of individual data elements that the system
  must provide, store, send, access, receive, etc., such as:

  \begin{enumerate}
  \itemsep1pt\parskip0pt\parsep0pt
  \item
    Names/identifiers

    \begin{enumerate}
    \itemsep1pt\parskip0pt\parsep0pt
    \item
      Project-unique identifier
    \item
      Non-technical (natural-language) name
    \item
      DoD standard data element name
    \item
      Technical name (e.g., variable or field name in code or database)
    \item
      Abbreviation or synonymous names
    \end{enumerate}
  \item
    Data type (alphanumeric, integer, etc.)
  \item
    Size and format (such as length and punctuation of a character
    string)
  \item
    Units of measurement (such as meters, dollars, nanoseconds)
  \item
    Range or enumeration of possible values (such as 0-99)
  \item
    Accuracy (how correct) and precision (number of significant digits)
  \item
    Priority, timing, frequency, volume, sequencing, and other
    constraints, such as whether the data element may be updated and
    whether business rules apply
  \item
    Security and privacy constraints
  \item
    Sources (setting/sending entities) and recipients (using/receiving
    entities)
  \end{enumerate}
\item
  Required characteristics of data element assemblies (records,
  messages, files, arrays, displays, reports, etc.) that the system must
  provide, store, send, access, receive, etc., such as:

  \begin{enumerate}
  \itemsep1pt\parskip0pt\parsep0pt
  \item
    Names/identifiers

    \begin{enumerate}
    \itemsep1pt\parskip0pt\parsep0pt
    \item
      Project-unique identifier
    \item
      Non-technical (natural language) name
    \item
      Technical name (e.g., record or data structure name in code or
      database)
    \item
      Abbreviations or synonymous names
    \end{enumerate}
  \item
    Data elements in the assembly and their structure (number, order,
    grouping)
  \item
    Medium (such as disk) and structure of data elements/assemblies on
    the medium
  \item
    Visual and auditory characteristics of displays and other outputs
    (such as colors, layouts, fonts, icons and other display elements,
    beeps, lights)
  \item
    Relationships among assemblies, such as sorting/access
    characteristics
  \item
    Priority, timing, frequency, volume, sequencing, and other
    constraints, such as whether the assembly may be updated and whether
    business rules apply
  \item
    Security and privacy constraints
  \item
    Sources (setting/sending entities) and recipients (using/receiving
    entities)
  \end{enumerate}
\item
  Required characteristics of communication methods that the system must
  use for the interface, such as:

  \begin{enumerate}
  \itemsep1pt\parskip0pt\parsep0pt
  \item
    Project-unique identifier(s)
  \item
    Communication links/bands/frequencies/media and their
    characteristics
  \item
    Message formatting
  \item
    Flow control (such as sequence numbering and buffer allocation)
  \item
    Data transfer rate, whether periodic/aperiodic, and interval between
    transfers
  \item
    Routing, addressing, and naming conventions
  \item
    Transmission services, including priority and grade
  \item
    Safety/security/privacy considerations, such as encryption, user
    authentication, compartmentalization, and auditing
  \end{enumerate}
\item
  Required characteristics of protocols the system must use for the
  interface, such as:

  \begin{enumerate}
  \itemsep1pt\parskip0pt\parsep0pt
  \item
    Project-unique identifier(s)
  \item
    Priority/layer of the protocol
  \item
    Packeting, including fragmentation and reassembly, routing, and
    addressing
  \item
    Legality checks, error control, and recovery procedures
  \item
    Synchronization, including connection establishment, maintenance,
    termination
  \item
    Status, identification, and any other reporting features
  \end{enumerate}
\item
  Other required characteristics, such as physical compatibility of the
  interfacing entities (dimensions, tolerances, loads, plug
  compatibility, etc.), voltages, etc.
\end{enumerate}

\subsection{System internal interface requirements}

This paragraph shall specify the requirements, if any, imposed on
interfaces internal to the system. If all internal interfaces are left
to the design or to requirement specifications for system components,
this fact shall be so stated. If such requirements are to be imposed,
paragraph 3.3 of this DID provides a list of topics to be considered.

\subsection{System internal data requirements}

This paragraph shall specify the requirements, if any, imposed on data
internal to the system. Included shall be requirements, if any, on
databases and data files to be included in the system. If all decisions
about internal data are left to the design or to requirements
specifications for system components, this fact shall be so stated. If
such requirements are to be imposed, paragraphs 3.3.x.c and 3.3.x.d of
this DID provide a list of topics to be considered.

\subsection{Adaptation requirements}

This paragraph shall specify the requirements, if any, concerning
installation-dependent data that the system is required to provide (such
as site-dependent latitude and longitude or site-dependent state tax
codes) and operational parameters that the system is required to use
that may vary according to operational needs (such as parameters
indicating operation-dependent targeting constants or data recording).

\subsection{Safety requirements}

This paragraph shall specify the system requirements, if any, concerned
with preventing or minimizing unintended hazards to personnel, property,
and the physical environment. Examples include restricting the use of
dangerous materials; classifying explosives for purposes of shipping,
handling, and storing; abort/escape provisions from enclosures; gas
detection and warning devices; grounding of electrical systems;
decontamination; and explosion proofing. This paragraph shall include
the system requirements, if any, for nuclear components, including, as
applicable, requirements for component design, prevention of inadvertent
detonation, and compliance with nuclear safety rules.

\subsection{Security and privacy requirements}

This paragraph shall specify the system requirements, if any, concerned
with maintaining security and privacy. The requirements shall include,
as applicable, the security/privacy environment in which the system must
operate, the type and degree of security or privacy to be provided, the
security/privacy risks the system must withstand, required safeguards to
reduce those risks, the security/privacy policy that must be met, the
security/privacy accountability the system must provide, and the
criteria that must be met for security/privacy
certification/accreditation.

\subsection{System environment requirements}

This paragraph shall specify the requirements, if any, regarding the
environment in which the system must operate. Examples for a software
system are the computer hardware and operating system on which the
software must run. (Additional requirements concerning computer
resources are given in the next paragraph). Examples for a
hardware-software system include the environmental conditions that the
system must withstand during transportation, storage, and operation,
such as conditions in the natural environment (wind, rain, temperature,
geographic location), the induced environment (motion, shock, noise,
electromagnetic radiation), and environments due to enemy action
(explosions, radiation).

\subsection{Computer resource requirements}

This paragraph shall be divided into the following subparagraphs.
Depending upon the nature of the system, the computer resources covered
in these subparagraphs may constitute the environment of the system (as
for a software system) or components of the system (as for a
hardware-software system).

\subsubsection{Computer hardware requirements}

This paragraph shall specify the requirements, if any, regarding
computer hardware that must be used by, or incorporated into, the
system. The requirements shall include, as applicable, number of each
type of equipment, type, size, capacity, and other required
characteristics of processors, memory, input/output devices, auxiliary
storage, communications/network equipment, and other required equipment.

\subsubsection{Computer hardware resource utilization
requirements}

This paragraph shall specify the requirements, if any, on the system's
computer hardware resource utilization, such as maximum allowable use of
processor capacity, memory capacity, input/output device capacity,
auxiliary storage device capacity, and communications/network equipment
capacity. The requirements (stated, for example, as percentages of the
capacity of each computer hardware resource) shall include the
conditions, if any, under which the resource utilization is to be
measured.

\subsubsection{Computer software requirements}

This paragraph shall specify the requirements, if any, regarding
computer software that must be used by, or incorporated into, the
system. Examples include operating systems, database management systems,
communications/ network software, utility software, input and equipment
simulators, test software, and manufacturing software. The correct
nomenclature, version, and documentation references of each such
software item shall be provided.

\subsubsection{Computer communications requirements}

This paragraph shall specify the additional requirements, if any,
concerning the computer communications that must be used by, or
incorporated into, the system. Examples include geographic locations to
be linked; configuration and network topology; transmission techniques;
data transfer rates; gateways; required system use times; type and
volume of data to be transmitted/received; time boundaries for
transmission/reception/response; peak volumes of data; and diagnostic
features.

\subsection{System quality factors}

This paragraph shall specify the requirements, if any, pertaining to
system quality factors. Examples include quantitative requirements
concerning system functionality (the ability to perform all required
functions), reliability (the ability to perform with correct, consistent
results -- such as mean time between failure for equipment),
maintainability (the ability to be easily serviced, repaired, or
corrected), availability (the ability to be accessed and operated when
needed), flexibility (the ability to be easily adapted to changing
requirements), portability of software (the ability to be easily
modified for a new environment), reusability (the ability to be used in
multiple applications), testability (the ability to be easily and
thoroughly tested), usability (the ability to be easily learned and
used), and other attributes.

\subsection{Design and construction constraints}

This paragraph shall specify the requirements, if any, that constrain
the design and construction of the system. For hardware-software
systems, this paragraph shall include the physical requirements imposed
on the system. These requirements may be specified by reference to
appropriate commercial or military standards and specifications.
Examples include requirements concerning:

\begin{enumerate}
\itemsep1pt\parskip0pt\parsep0pt
\item
  Use of a particular system architecture or requirements on the
  architecture, such as required subsystems; use of standard, military,
  or existing components; or use of Government/acquirer-furnished
  property (equipment, information, or software)
\item
  Use of particular design or construction standards; use of particular
  data standards; use of a particular programming language; workmanship
  requirements and production techniques
\item
  Physical characteristics of the system (such as weight limits,
  dimensional limits, color, protective coatings); interchangeability of
  parts; ability to be transported from one location to another; ability
  to be carried or set up by one, or a given number of, persons
\item
  Materials that can and cannot be used; requirements on the handling of
  toxic materials; limits on the electromagnetic radiation that the
  system is permitted to generate
\item
  Use of nameplates, part marking, serial and lot number marking, and
  other identifying markings
\item
  Flexibility and expandability that must be provided to support
  anticipated areas of growth or changes in technology, threat, or
  mission
\end{enumerate}

\subsection{Personnel-related requirements}

This paragraph shall specify the system requirements, if any, included
to accommodate the number, skill levels, duty cycles, training needs, or
other information about the personnel who will use or support the
system. Examples include requirements for the number of work stations to
be provided and for built-in help and training features. Also included
shall be the human factors engineering requirements, if any, imposed on
the system. These requirements shall include, as applicable,
considerations for the capabilities and limitations of humans,
foreseeable human errors under both normal and extreme conditions, and
specific areas where the effects of human error would be particularly
serious. Examples include requirements for adjustable-height work
stations, color and duration of error messages, physical placement of
critical indicators or buttons, and use of auditory signals.

\subsection{Training-related requirements}

This paragraph shall specify the system requirements, if any, pertaining
to training. Examples include training devices and training materials to
be included in the system.

\subsection{Logistics-related requirements}

This paragraph shall specify the system requirements, if any, concerned
with logistics considerations. These considerations may include: system
maintenance, software support, system transportation modes, supply
system requirements, impact on existing facilities, and impact on
existing equipment.

\subsection{Other requirements}

This paragraph shall specify additional system requirements, if any, not
covered in the previous paragraphs. Examples include requirements for
system documentation, such as specifications, drawings, technical
manuals, test plans and procedures, and installation instruction data,
if not covered in other contractual documents.

\subsection{Packaging requirements}

This section shall specify the requirements, if any, for packaging,
labeling, and handling the system and its components for delivery.
Applicable military specifications and standards may be referenced if
appropriate.

\subsection{Precedence and criticality of requirements}

This paragraph shall specify, if applicable, the order of precedence,
criticality, or assigned weights indicating the relative importance of
the requirements in this specification. Examples include identifying
those requirements deemed critical to safety, to security, or to privacy
for purposes of singling them out for special treatment. If all
requirements have equal weight, this paragraph shall so state.

\section{Qualification provisions}

This section shall define a set of qualification methods and shall
specify for each requirement in Section 3 the method(s) to be used to
ensure that the requirement has been met. A table may be used to present
this information, or each requirement in Section 3 may be annotated with
the method(s) to be used. Qualification methods may include:

\begin{enumerate}
\itemsep1pt\parskip0pt\parsep0pt
\item
  Demonstration: The operation of the system, or a part of the system,
  that relies on observable functional operation not requiring the use
  of instrumentation, special test equipment, or subsequent analysis.
\item
  Test: The operation of the system, or a part of the system, using
  instrumentation or other special test equipment to collect data for
  later analysis.
\item
  Analysis: The processing of accumulated data obtained from other
  qualification methods. Examples are reduction, interpolation, or
  extrapolation of test results.
\item
  Inspection: The visual examination of system components,
  documentation, etc.
\item
  Special qualification methods. Any special qualification methods for
  the system, such as special tools, techniques, procedures, facilities,
  acceptance limits, use of standard samples, preproduction or periodic
  production samples, pilot models, or pilot lots.
\end{enumerate}

\section{Requirements traceability}

For system-level specifications, this paragraph does not apply. For
subsystem-level specifications, this paragraph shall contain:

\begin{enumerate}
\itemsep1pt\parskip0pt\parsep0pt
\item
  Traceability from each subsystem requirement in this specification to
  the system requirements it addresses. (Alternatively, this
  traceability may be provided by annotating each requirement in Section
  3.)

  Note: Each level of system refinement may result in
  requirements not directly traceable to higher-level requirements. For
  example, a system architectural design that creates two subsystems may
  result in requirements about how the subsystems will interface, even
  though these interfaces are not covered in system requirements. Such
  requirements may be traced to a general requirement such as ``system
  implementation'' or to the system design decisions that resulted in
  their generation.
\item
  Traceability from each system requirement that has been allocated to
  the subsystem covered by this specification to the subsystem
  requirements that address it. All system requirements allocated to the
  subsystem shall be accounted for. Those that trace to subsystem
  requirements contained in IRSs shall reference those IRSs.
\end{enumerate}

\section{Notes}

This section shall contain any general information that aids in
understanding this document (e.g., background information, glossary,
rationale). This section shall contain an alphabetical listing of all
acronyms, abbreviations, and their meanings as used in this document and
a list of any terms and definitions needed to understand this document.

\appendix

\section{Appendixes}

Appendixes may be used to provide information published separately for
convenience in document maintenance (e.g., charts, classified data). As
applicable, each appendix shall be referenced in the main body of the
document where the data would normally have been provided. Appendixes
may be bound as separate documents for ease in handling. Appendixes
shall be lettered alphabetically (A, B, etc.).

\end{document}
