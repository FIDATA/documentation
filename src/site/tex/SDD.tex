\documentclass{fidata-report-template}

\begin{document}

\frontmatter

\title{Software Design Description}

\date{2014-02-24}

\author{Basil Peace}

\maketitle
\tableofcontents
% \listoffiguresandtables

\section{Scope}

This section shall be divided into the following paragraphs.

\subsection{Identification}

This paragraph shall contain a full identification of the system and the
software to which this document applies, including, as applicable,
identification number(s), title(s), abbreviation(s), version number(s),
and release number(s).

\subsection{System overview}

This paragraph shall briefly state the purpose of the system and the
software to which this document applies. It shall describe the general
nature of the system and software; summarize the history of system
development, operation, and maintenance; identify the project sponsor,
acquirer, user, developer, and support agencies; identify current and
planned operating sites; and list other relevant documents.

\subsection{Document overview}

This paragraph shall summarize the purpose and contents of this document
and shall describe any security or privacy considerations associated
with its use.

\section{}

\section{Referenced documents}

This section shall list the number, title, revision, and date of all
documents referenced in this document. This section shall also identify
the source for all documents not available through normal Government
stocking activities.

\section{CSCI-wide design decisions}

This section shall be divided into paragraphs as needed to present
CSCI-wide design decisions, that is, decisions about the CSCI's
behavioral design (how it will behave, from a user's point of view, in
meeting its requirements, ignoring internal implementation) and other
decisions affecting the selection and design of the software units that
make up the CSCI. If all such decisions are explicit in the CSCI
requirements or are deferred to the design of the CSCI's software units,
this section shall so state. Design decisions that respond to
requirements designated critical, such as those for safety, security, or
privacy, shall be placed in separate subparagraphs. If a design decision
depends upon system states or modes, this dependency shall be indicated.
Design conventions needed to understand the design shall be presented or
referenced. Examples of CSCI-wide design decisions are the following:

\begin{enumerate}
\itemsep1pt\parskip0pt\parsep0pt
\item
  Design decisions regarding inputs the CSCI will accept and outputs it
  will produce, including interfaces with other systems, HWCIs, CSCIs,
  and users (4.3.x of this DID identifies topics to be considered in
  this description). If part or all of this information is given in
  Interface Design Descriptions (IDDs), they may be referenced.
\item
  Design decisions on CSCI behavior in response to each input or
  condition, including actions the CSCI will perform, response times and
  other performance characteristics, description of physical systems
  modeled, selected equations/algorithms/rules, and handling of
  unallowed inputs or conditions.
\item
  Design decisions on how databases/data files will appear to the user
  (4.3.x of this DID identifies topics to be considered in this
  description). If part or all of this information is given in Database
  Design Descriptions (DBDDs), they may be referenced.
\item
  Selected approach to meeting safety, security, and privacy
  requirements.
\item
  Other CSCI-wide design decisions made in response to requirements,
  such as selected approach to providing required flexibility,
  availability, and maintainability.
\end{enumerate}

\section{CSCI architectural design}

This section shall be divided into the following paragraphs to describe
the CSCI architectural design. If part or all of the design depends upon
system states or modes, this dependency shall be indicated. If design
information falls into more than one paragraph, it may be presented once
and referenced from the other paragraphs. Design conventions needed to
understand the design shall be presented or referenced.

\subsection{CSCI components}

This paragraph shall:

\begin{enumerate}
\itemsep1pt\parskip0pt\parsep0pt
\item
  Identify the software units that make up the CSCI. Each software unit
  shall be assigned a project-unique identifier.

  Note: A software
  unit is an element in the design of a CSCI; for example, a major
  subdivision of a CSCI, a component of that subdivision, a class,
  object, module, function, routine, or database. Software units may
  occur at different levels of a hierarchy and may consist of other
  software units. Software units in the design may or may not have a
  one-to-one relationship with the code and data entities (routines,
  procedures, databases, data files, etc.) that implement them or with
  the computer files containing those entities. A database may be
  treated as a CSCI or as a software unit. The SDD may refer to software
  units by any name(s) consistent with the design methodology being
  used.
\item
  Show the static (such as ``consists of'') relationship(s) of the
  software units. Multiple relationships may be presented, depending on
  the selected software design methodology (for example, in an
  object-oriented design, this paragraph may present the class and
  object structures as well as the module and process architectures of
  the CSCI).
\item
  State the purpose of each software unit and identify the CSCI
  requirements and CSCI-wide design decisions allocated to it.
  (Alternatively, the allocation of requirements may be provided in
  6.a.)
\item
  Identify each software unit's development status/type (such as new
  development, existing design or software to be reused as is, existing
  design or software to be reengineered, software to be developed for
  reuse, software planned for Build N, etc.) For existing design or
  software, the description shall provide identifying information, such
  as name, version, documentation references, library, etc.
\item
  Describe the CSCI's (and as applicable, each software unit's) planned
  utilization of computer hardware resources (such as processor
  capacity, memory capacity, input/output device capacity, auxiliary
  storage capacity, and communications/network equipment capacity). The
  description shall cover all computer hardware resources included in
  resource utilization requirements for the CSCI, in system-level
  resource allocations affecting the CSCI, and in resource utilization
  measurement planning in the Software Development Plan. If all
  utilization data for a given computer hardware resource are presented
  in a single location, such as in one SDD, this paragraph may reference
  that source. Included for each computer hardware resource shall be:

  \begin{enumerate}
  \itemsep1pt\parskip0pt\parsep0pt
  \item
    The CSCI requirements or system-level resource allocations being
    satisfied
  \item
    The assumptions and conditions on which the utilization data are
    based (for example, typical usage, worst-case usage, assumption of
    certain events)
  \item
    Any special considerations affecting the utilization (such as use of
    virtual memory, overlays, or multiprocessors or the impacts of
    operating system overhead, library software, or other implementation
    overhead)
  \item
    The units of measure used (such as percentage of processor capacity,
    cycles per second, bytes of memory, kilobytes per second)
  \item
    The level(s) at which the estimates or measures will be made (such
    as software unit, CSCI, or executable program)
  \end{enumerate}
\item
  Identify the program library in which the software that implements
  each software unit is to be placed
\end{enumerate}

\subsection{Concept of execution}

This paragraph shall describe the concept of execution among the
software units. It shall include diagrams and descriptions showing the
dynamic relationship of the software units, that is, how they will
interact during CSCI operation, including, as applicable, flow of
execution control, data flow, dynamically controlled sequencing, state
transition diagrams, timing diagrams, priorities among units, handling
of interrupts, timing/sequencing relationships, exception handling,
concurrent execution, dynamic allocation/deallocation, dynamic
creation/deletion of objects, processes, tasks, and other aspects of
dynamic behavior.

\subsection{Interface design}

This paragraph shall be divided into the following subparagraphs to
describe the interface characteristics of the software units. It shall
include both interfaces among the software units and their interfaces
with external entities such as systems, configuration items, and users.
If part or all of this information is contained in Interface Design
Descriptions (IDDs), in section 5 of the SDD, or elsewhere, these
sources may be referenced.

\subsubsection{Interface identification and diagrams}

This paragraph shall state the project-unique identifier assigned to
each interface and shall identify the interfacing entities (software
units, systems, configuration items, users, etc.) by name, number,
version, and documentation references, as applicable. The identification
shall state which entities have fixed interface characteristics (and
therefore impose interface requirements on interfacing entities) and
which are being developed or modified (thus having interface
requirements imposed on them). One or more interface diagrams shall be
provided, as appropriate, to depict the interfaces.

\subsubsection{(Project unique identifier of interface)}

This paragraph (beginning with 4.3.2) shall identify an interface by
project unique identifier, shall briefly identify the interfacing
entities, and shall be divided into subparagraphs as needed to describe
the interface characteristics of one or both of the interfacing
entities. If a given interfacing entity is not covered by this SDD (for
example, an external system) but its interface characteristics need to
be mentioned to describe interfacing entities that are, these
characteristics shall be stated as assumptions or as ``When {[}the
entity not covered{]} does this, {[}the entity that is covered{]} will .
. . .'' This paragraph may reference other documents (such as data
dictionaries, standards for protocols, and standards for user
interfaces) in place of stating the information here. The design
description shall include the following, as applicable, presented in any
order suited to the information to be provided, and shall note any
differences in these characteristics from the point of view of the
interfacing entities (such as different expectations about the size,
frequency, or other characteristics of data elements):

\begin{enumerate}
\itemsep1pt\parskip0pt\parsep0pt
\item
  Priority assigned to the interface by the interfacing entity(ies)
\item
  Type of interface (such as real-time data transfer,
  storage-and-retrieval of data, etc.) to be implemented
\item
  Characteristics of individual data elements that the interfacing
  entity(ies) will provide, store, send, access, receive, etc., such as:

  \begin{enumerate}
  \itemsep1pt\parskip0pt\parsep0pt
  \item
    Names/identifiers

    \begin{enumerate}
    \itemsep1pt\parskip0pt\parsep0pt
    \item
      Project-unique identifier
    \item
      Non-technical (natural-language) name
    \item
      DoD standard data element name
    \item
      Technical name (e.g., variable or field name in code or database)
    \item
      Abbreviation or synonymous names
    \end{enumerate}
  \item
    Data type (alphanumeric, integer, etc.)
  \item
    Size and format (such as length and punctuation of a character
    string)
  \item
    Units of measurement (such as meters, dollars, nanoseconds)
  \item
    Range or enumeration of possible values (such as 0-99)
  \item
    Accuracy (how correct) and precision (number of significant digits)
  \item
    Priority, timing, frequency, volume, sequencing, and other
    constraints, such as whether the data element may be updated and
    whether business rules apply
  \item
    Security and privacy constraints
  \item
    Sources (setting/sending entities) and recipients (using/receiving
    entities)
  \end{enumerate}
\item
  Characteristics of data element assemblies (records, messages, files,
  arrays, displays, reports, etc.) that the interfacing entity(ies) will
  provide, store, send, access, receive, etc., such as:

  \begin{enumerate}
  \itemsep1pt\parskip0pt\parsep0pt
  \item
    Names/identifiers

    \begin{enumerate}
    \itemsep1pt\parskip0pt\parsep0pt
    \item
      Project-unique identifier
    \item
      Non-technical (natural language) name
    \item
      Technical name (e.g., record or data structure name in code or
      database)
    \item
      Abbreviations or synonymous names
    \end{enumerate}
  \item
    Data elements in the assembly and their structure (number, order,
    grouping)
  \item
    Medium (such as disk) and structure of data elements/assemblies on
    the medium
  \item
    Visual and auditory characteristics of displays and other outputs
    (such as colors, layouts, fonts, icons and other display elements,
    beeps, lights)
  \item
    Relationships among assemblies, such as sorting/access
    characteristics
  \item
    Priority, timing, frequency, volume, sequencing, and other
    constraints, such as whether the assembly may be updated and whether
    business rules apply
  \item
    Security and privacy constraints
  \item
    Sources (setting/sending entities) and recipients (using/receiving
    entities)
  \end{enumerate}
\item
  Characteristics of communication methods that the interfacing
  entity(ies) will use for the interface, such as:

  \begin{enumerate}
  \itemsep1pt\parskip0pt\parsep0pt
  \item
    Project-unique identifier(s)
  \item
    Communication links/bands/frequencies/media and their
    characteristics
  \item
    Message formatting
  \item
    Flow control (such as sequence numbering and buffer allocation)
  \item
    Data transfer rate, whether periodic/aperiodic, and interval between
    transfers
  \item
    Routing, addressing, and naming conventions
  \item
    Transmission services, including priority and grade
  \item
    Safety/security/privacy considerations, such as encryption, user
    authentication, compartmentalization, and auditing
  \end{enumerate}
\item
  Characteristics of protocols that the interfacing entity(ies) will use
  for the interface, such as:

  \begin{enumerate}
  \itemsep1pt\parskip0pt\parsep0pt
  \item
    Project-unique identifier(s)
  \item
    Priority/layer of the protocol
  \item
    Packeting, including fragmentation and reassembly, routing, and
    addressing
  \item
    Legality checks, error control, and recovery procedures
  \item
    Synchronization, including connection establishment, maintenance,
    termination
  \item
    Status, identification, and any other reporting features
  \end{enumerate}
\item
  Other characteristics, such as physical compatibility of the
  interfacing entity(ies) (dimensions, tolerances, loads, voltages, plug
  compatibility, etc.)
\end{enumerate}

\section{CSCI detailed design}

This section shall be divided into the following paragraphs to describe
each software unit of the CSCI. If part of all of the design depends
upon system states or modes, this dependency shall be indicated. If
design information falls into more than one paragraph, it may be
presented once and referenced from the other paragraphs. Design
conventions needed to understand the design shall be presented or
referenced. Interface characteristics of software units may be described
here, in Section 4, or in Interface Design Descriptions (IDDs). Software
units that are databases, or that are used to access or manipulate
databases, may be described here or in Database Design Descriptions
(DBDDs).

\subsection{(Project-unique identifier of a software unit, or
designator of a group of software units)}

This paragraph shall identify a software unit by project-unique
identifier and shall describe the unit. The description shall include
the following information, as applicable. Alternatively, this paragraph
may designate a group of software units and identify and describe the
software units in subparagraphs. Software units that contain other
software units may reference the descriptions of those units rather than
repeating information.

\begin{enumerate}
\itemsep1pt\parskip0pt\parsep0pt
\item
  Unit design decisions, if any, such as algorithms to be used, if not
  previously selected
\item
  Any constraints, limitations, or unusual features in the design of the
  software unit
\item
  The programming language to be used and rationale for its use if other
  than the specified CSCI language
\item
  If the software unit consists of or contains procedural commands (such
  as menu selections in a database management system (DBMS) for defining
  forms and reports, on-line DBMS queries for database access and
  manipulation, input to a graphical user interface (GUI) builder for
  automated code generation, commands to the operating system, or shell
  scripts), a list of the procedural commands and reference to user
  manuals or other documents that explain them
\item
  If the software unit contains, receives, or outputs data, a
  description of its inputs, outputs, and other data elements and data
  element assemblies, as applicable. Paragraph 4.3.x of this DID
  provides a list of topics to be covered, as applicable. Data local to
  the software unit shall be described separately from data input to or
  output from the software unit. If the software unit is a database, a
  corresponding Database Design Description (DBDD) shall be referenced;
  interface characteristics may be provided here or by referencing
  section 4 or the corresponding Interface Design Description(s).
\item
  If the software unit contains logic, the logic to be used by the
  software unit, including, as applicable:

  \begin{enumerate}
  \itemsep1pt\parskip0pt\parsep0pt
  \item
    Conditions in effect within the software unit when its execution is
    initiated
  \item
    Conditions under which control is passed to other software units
  \item
    Response and response time to each input, including data conversion,
    renaming, and data transfer operations
  \item
    Sequence of operations and dynamically controlled sequencing during
    the software unit's operation, including:

    \begin{enumerate}
    \itemsep1pt\parskip0pt\parsep0pt
    \item
      The method for sequence control
    \item
      The logic and input conditions of that method, such as timing
      variations, priority assignments
    \item
      Data transfer in and out of memory
    \item
      The sensing of discrete input signals, and timing relationships
      between interrupt operations within the software unit
    \end{enumerate}
  \item
    Exception and error handling
  \end{enumerate}
\end{enumerate}

\section{Requirements traceability}

This section shall contain:

\begin{enumerate}
\itemsep1pt\parskip0pt\parsep0pt
\item
  Traceability from each software unit identified in this SDD to the
  CSCI requirements allocated to it. (Alternatively, this traceability
  may be provided in 4.1.)
\item
  Traceability from each CSCI requirement to the software units to which
  it is allocated.
\end{enumerate}

\section{Notes}

This section shall contain any general information that aids in
understanding this document (e.g., background information, glossary,
rationale). This section shall include an alphabetical listing of all
acronyms, abbreviations, and their meanings as used in this document and
a list of any terms and definitions needed to understand this document.

\appendix

\section{Appendixes}

Appendixes may be used to provide information published separately for
convenience in document maintenance (e.g., charts, classified data). As
applicable, each appendix shall be referenced in the main body of the
document where the data would normally have been provided. Appendixes
may be bound as separate documents for ease in handling. Appendixes
shall be lettered alphabetically (A, B, etc.).

\end{document}
