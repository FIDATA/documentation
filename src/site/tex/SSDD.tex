\documentclass{fidata-report-template}

\begin{document}

\frontmatter

\title{System/Subsystem Design Description}

\date{2014-02-24}

\author{Basil Peace}

\maketitle
\tableofcontents
% \listoffiguresandtables

\section{Scope}

This section shall be divided into the following paragraphs.

\subsection{Identification}

This paragraph shall contain a full identification of the system to
which this document applies, including, as applicable, identification
number(s), title(s), abbreviation(s), version number(s), and release
number(s).

\subsection{System overview}

This paragraph shall briefly state the purpose of the system to which
this document applies. It shall describe the general nature of the
system; summarize the history of system development, operation, and
maintenance; identify the project sponsor, acquirer, user, developer,
and support agencies; identify current and planned operating sites; and
list other relevant documents.

\subsection{Document overview}

paragraph shall summarize the purpose and contents of this document and
shall describe any security or privacy considerations associated with
its use.

\section{Referenced documents}

This section shall list the number, title, revision, and date of all
documents referenced in this document. This section shall also identify
the source for all documents not available through normal Government
stocking activities.

\section{System-wide design decisions}

This section shall be divided into paragraphs as needed to present
system-wide design decisions, that is, decisions about the system's
behavioral design (how it will behave, from a user's point of view, in
meeting its requirements, ignoring internal implementation) and other
decisions affecting the selection and design of system components. If
all such decisions are explicit in the requirements or are deferred to
the design of the system components, this section shall so state. Design
decisions that respond to requirements designated critical, such as
those for safety, security, or privacy, shall be placed in separate
subparagraphs. If a design decision depends upon system states or modes,
this dependency shall be indicated. Design conventions needed to
understand the design shall be presented or referenced. Examples of
system-wide design decisions are the following:

\begin{enumerate}
\itemsep1pt\parskip0pt\parsep0pt
\item
  Design decisions regarding inputs the system will accept and outputs
  it will produce, including interfaces with other systems,
  configuration items, and users (4.3.x of this DID identifies topics to
  be considered in this description). If part or all of this information
  is given in Interface Design Descriptions (IDDs), they may be
  referenced.
\item
  Design decisions on system behavior in response to each input or
  condition, including actions the system will perform, response times
  and other performance characteristics, description of physical systems
  modeled, selected equations/algorithms/ rules, and handling of
  unallowed inputs or conditions.
\item
  Design decisions on how system databases/data files will appear to the
  user (4.3.x of this DID identifies topics to be considered in this
  description). If part or all of this information is given in Database
  Design Descriptions (DBDDs), they may be referenced.
\item
  Selected approach to meeting safety, security, and privacy
  requirements.
\item
  Design and construction choices for hardware or hardware-software
  systems, such as physical size, color, shape, weight, materials, and
  markings.
\item
  Other system-wide design decisions made in response to requirements,
  such as selected approach to providing required flexibility,
  availability, and maintainability.
\end{enumerate}

\section{System architectural design}

This section shall be divided into the following paragraphs to describe
the system architectural design. If part or all of the design depends
upon system states or modes, this dependency shall be indicated. If
design information falls into more than one paragraph, it may be
presented once and referenced from the other paragraphs. Design
conventions needed to understand the design shall be presented or
referenced.

Note: For brevity, this section is written in terms of
organizing a system directly into Hardware Configuration Items (HWCIs),
Computer Software Configuration Items (CSCIs), and manual operations,
but should be interpreted to cover organizing a system into subsystems,
organizing a subsystem into HWCIs, CSCIs, and manual operations, or
other variations as appropriate.

\subsection{System components}

This paragraph shall:

\begin{enumerate}
\itemsep1pt\parskip0pt\parsep0pt
\item
  Identify the components of the system (HWCIs, CSCIs, and manual
  operations). Each component shall be assigned a project-unique
  identifier. Note: a database may be treated as a CSCI or as part of a
  CSCI.
\item
  Show the static (such as ``consists of'') relationship(s) of the
  components. Multiple relationships may be presented, depending on the
  selected design methodology.
\item
  State the purpose of each component and identify the system
  requirements and system-wide design decisions allocated to it.
  (Alternatively, the allocation of requirements may be provided in
  5.a.)
\item
  Identify each component's development status/type, if known (such as
  new development, existing component to be reused as is, existing
  design to be reused as is, existing design or component to be
  reengineered, component to be developed for reuse, component planned
  for Build N, etc.) For existing design or components, the description
  shall provide identifying information, such as name, version,
  documentation references, location, etc.
\item
  For each computer system or other aggregate of computer hardware
  resources identified for use in the system, describe its computer
  hardware resources (such as processors, memory, input/output devices,
  auxiliary storage, and communications/ network equipment). Each
  description shall, as applicable, identify the configuration items
  that will use the resource, describe the allocation of resource
  utilization to each CSCI that will use the resource (for example, 20\%
  of the resource's capacity allocated to CSCI 1, 30\% to CSCI 2),
  describe the conditions under which utilization will be measured, and
  describe the characteristics of the resource:

  \begin{enumerate}
  \itemsep1pt\parskip0pt\parsep0pt
  \item
    Descriptions of computer processors shall include, as applicable,
    manufacturer name and model number, processor speed/capacity,
    identification of instruction set architecture, applicable
    compiler(s), word size (number of bits in each computer word),
    character set standard (such as ASCII, EBCDIC), and interrupt
    capabilities.
  \item
    Descriptions of memory shall include, as applicable, manufacturer
    name and model number and memory size, type, speed, and
    configuration (such as 256K cache memory, 16MB RAM (4MB x 4)).
  \item
    Descriptions of input/output devices shall include, as applicable,
    manufacturer name and model number, type of device, and device
    speed/capacity.
  \item
    Descriptions of auxiliary storage shall include, as applicable,
    manufacturer name and model number, type of storage, amount of
    installed storage, and storage speed.
  \item
    Descriptions of communications/network equipment, such as modems,
    network interface cards, hubs, gateways, cabling, high speed data
    lines, or aggregates of these or other components, shall include, as
    applicable, manufacturer name and model number, data transfer
    rates/capacities, network topologies, transmission techniques, and
    protocols used.
  \item
    Each description shall also include, as applicable, growth
    capabilities, diagnostic capabilities, and any additional hardware
    capabilities relevant to the description.
  \end{enumerate}
\item
  Present a specification tree for the system, that is, a diagram that
  identifies and shows the relationships among the planned
  specifications for the system components.
\end{enumerate}

\subsection{Concept of execution}

This paragraph shall describe the concept of execution among the system
components. It shall include diagrams and descriptions showing the
dynamic relationship of the components, that is, how they will interact
during system operation, including, as applicable, flow of execution
control, data flow, dynamically controlled sequencing, state transition
diagrams, timing diagrams, priorities among components, handling of
interrupts, timing/sequencing relationships, exception handling,
concurrent execution, dynamic allocation/deallocation, dynamic
creation/deletion of objects, processes, tasks, and other aspects of
dynamic behavior.

\subsection{Interface design}

This paragraph shall be divided into the following subparagraphs to
describe the interface characteristics of the system components. It
shall include both interfaces among the components and their interfaces
with external entities such as other systems, configuration items, and
users. Note: There is no requirement for these interfaces to be
completely designed at this level; this paragraph is provided to allow
the recording of interface design decisions made as part of system
architectural design. If part or all of this information is contained in
Interface Design Descriptions (IDDs) or elsewhere, these sources may be
referenced.

\subsubsection{Interface identification and diagrams}

This paragraph shall state the project-unique identifier assigned to
each interface and shall identify the interfacing entities (systems,
configuration items, users, etc.) by name, number, version, and
documentation references, as applicable. The identification shall state
which entities have fixed interface characteristics (and therefore
impose interface requirements on interfacing entities) and which are
being developed or modified (thus having interface requirements imposed
on them). One or more interface diagrams shall be provided, as
appropriate, to depict the interfaces.

\subsubsection{(Project unique identifier of interface)}

This paragraph (beginning with 4.3.2) shall identify an interface by
project unique identifier, shall briefly identify the interfacing
entities, and shall be divided into subparagraphs as needed to describe
the interface characteristics of one or both of the interfacing
entities. If a given interfacing entity is not covered by this SSDD (for
example, an external system) but its interface characteristics need to
be mentioned to describe interfacing entities that are, these
characteristics shall be stated as assumptions or as ``When {[}the
entity not covered{]} does this, {[}the entity that is covered{]} will
\ldots{}.'' This paragraph may reference other documents (such as data
dictionaries, standards for protocols, and standards for user
interfaces) in place of stating the information here. The design
description shall include the following, as applicable, presented in any
order suited to the information to be provided, and shall note any
differences in these characteristics from the point of view of the
interfacing entities (such as different expectations about the size,
frequency, or other characteristics of data elements):

\begin{enumerate}
\itemsep1pt\parskip0pt\parsep0pt
\item
  Priority assigned to the interface by the interfacing entity(ies)
\item
  Type of interface (such as real-time data transfer,
  storage-and-retrieval of data, etc.) to be implemented
\item
  Characteristics of individual data elements that the interfacing
  entity(ies) will provide, store, send, access, receive, etc., such as:

  \begin{enumerate}
  \itemsep1pt\parskip0pt\parsep0pt
  \item
    Names/identifiers

    \begin{enumerate}
    \itemsep1pt\parskip0pt\parsep0pt
    \item
      Project-unique identifier
    \item
      Non-technical (natural-language) name
    \item
      DoD standard data element name
    \item
      Technical name (e.g., variable or field name in code or database)
    \item
      Abbreviation or synonymous names
    \end{enumerate}
  \item
    Data type (alphanumeric, integer, etc.)
  \item
    Size and format (such as length and punctuation of a character
    string)
  \item
    Units of measurement (such as meters, dollars, nanoseconds)
  \item
    Range or enumeration of possible values (such as 0-99)
  \item
    Accuracy (how correct) and precision (number of significant digits)
  \item
    Priority, timing, frequency, volume, sequencing, and other
    constraints, such as whether the data element may be updated and
    whether business rules apply
  \item
    Security and privacy constraints
  \item
    Sources (setting/sending entities) and recipients (using/receiving
    entities)
  \end{enumerate}
\item
  Sources (setting/sending entities) and recipients (using/receiving
  entities)

  \begin{enumerate}
  \itemsep1pt\parskip0pt\parsep0pt
  \item
    Names/identifiers

    \begin{enumerate}
    \itemsep1pt\parskip0pt\parsep0pt
    \item
      Project-unique identifier to be used for traceability
    \item
      Non-technical (natural language) name
    \item
      Technical name (e.g., record or data structure name in code or
      database)
    \item
      Abbreviations or synonymous names
    \end{enumerate}
  \item
    Data elements in the assembly and their structure (number, order,
    grouping)
  \item
    Medium (such as disk) and structure of data elements/assemblies on
    the medium
  \item
    Visual and auditory characteristics of displays and other outputs
    (such as colors, layouts, fonts, icons and other display elements,
    beeps, lights)
  \item
    Relationships among assemblies, such as sorting/access
    characteristics
  \item
    Priority, timing, frequency, volume, sequencing, and other
    constraints, such as whether the assembly may be updated and whether
    business rules apply
  \item
    Security and privacy constraints
  \item
    Sources (setting/sending entities) and recipients (using/receiving
    entities)
  \end{enumerate}
\item
  Characteristics of communication methods that the interfacing
  entity(ies) will use for the interface, such as:

  \begin{enumerate}
  \itemsep1pt\parskip0pt\parsep0pt
  \item
    Project-unique identifier(s)
  \item
    Communication links/bands/frequencies/media and their
    characteristics
  \item
    Message formatting
  \item
    Flow control (such as sequence numbering and buffer allocation)
  \item
    Data transfer rate, whether periodic/aperiodic, and interval between
    transfers
  \item
    Routing, addressing, and naming conventions
  \item
    Transmission services, including priority and grade
  \item
    Safety/security/privacy considerations, such as encryption, user
    authentication, compartmentalization, and auditing
  \end{enumerate}
\item
  Characteristics of protocols that the interfacing entity(ies) will use
  for the interface, such as:

  \begin{enumerate}
  \itemsep1pt\parskip0pt\parsep0pt
  \item
    Project-unique identifier(s)
  \item
    Priority/layer of the protocol
  \item
    Packeting, including fragmentation and reassembly, routing, and
    addressing
  \item
    Legality checks, error control, and recovery procedures
  \item
    Synchronization, including connection establishment, maintenance,
    termination
  \item
    Status, identification, and any other reporting features
  \end{enumerate}
\item
  Other characteristics, such as physical compatibility of the
  interfacing entity(ies) (dimensions, tolerances, loads, voltages, plug
  compatibility, etc.)
\end{enumerate}

\section{Requirements traceability}

This paragraph shall contain:

\begin{enumerate}
\itemsep1pt\parskip0pt\parsep0pt
\item
  Traceability from each system component identified in this SSDD to the
  system requirements allocated to it. (Alternatively, this traceability
  may be provided in 4.1.)
\item
  Traceability from each system requirement to the system components to
  which it is allocated.
\end{enumerate}

\section{Notes}

This section shall contain any general information that aids in
understanding this document (e.g., background information, glossary,
rationale). This section shall contain an alphabetical listing of all
acronyms, abbreviations, and their meanings as used in this document and
a list of any terms and definitions needed to understand this document.

\appendix

\section{Appendixes}

Appendixes may be used to provide information published separately for
convenience in document maintenance (e.g., charts, classified data). As
applicable, each appendix shall be referenced in the main body of the
document where the data would normally have been provided. Appendixes
may be bound as separate documents for ease in handling. Appendixes
shall be lettered alphabetically (A, B, etc.).

\end{document}
