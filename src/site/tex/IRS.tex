\documentclass{fidata-report-template}

\begin{document}

\section{Scope}

This section shall be divided into the following paragraphs.

\subsection{Identification}

This paragraph shall contain a full identification of the systems, the
interfacing entities, and the interfaces to which this document applies,
including, as applicable, identification number(s), title(s),
abbreviation(s), version number(s), and release number(s).

\subsection{System overview}

This paragraph shall briefly state the purpose of the system(s) and
software to which this document applies. It shall describe the general
nature of the system and software; summarize the history of system
development, operation, and maintenance; identify the project sponsor,
acquirer, user, developer, and support agencies; identify current and
planned operating sites; and list other relevant documents.

\subsection{Document overview}

This paragraph shall summarize the purpose and contents of this document
and shall describe any security or privacy considerations associated
with its use.

\section{Referenced documents}

This section shall list the number, title, revision, and date of all
documents referenced in this specification. This section shall also
identify the source for all documents not available through normal
Government stocking activities.

\section{Requirements}

This section shall be divided into the following paragraphs to specify
the requirements imposed on one or more systems, subsystems,
configuration items, manual operations, or other system components to
achieve one or more interfaces among these entities. Each requirement
shall be assigned a project-unique identifier to support testing and
traceability and shall be stated in such a way that an objective test
can be defined for it. Each requirement shall be annotated with
associated qualification method(s) (see section 4) and traceability to
system (or subsystem, if applicable) requirements (see section 5.a) if
not provided in those sections. The degree of detail to be provided
shall be guided by the following rule: Include those characteristics of
the interfacing entities that are conditions for their acceptance; defer
to design descriptions those characteristics that the acquirer is
willing to leave up to the developer. If a given requirement fits into
more than one paragraph, it may be stated once and referenced from the
other paragraphs. If an interfacing entity included in this
specification will operate in states and/or modes having interface
requirements different from other states and modes, each requirement or
group of requirements for that entity shall be correlated to the states
and modes. The correlation may be indicated by a table or other method
in this paragraph, in an appendix referenced from this paragraph, or by
annotation of the requirements in the paragraphs where they appear.

\subsection{Interface identification and diagrams}

For each interface identified in 1.1, this paragraph shall include a
project-unique identifier and shall designate the interfacing entities
(systems, configuration items, users, etc.) by name, number, version,
and documentation references, as applicable. The identification shall
state which entities have fixed interface characteristics (and therefore
impose interface requirements on interfacing entities) and which are
being developed or modified (thus having interface requirements imposed
on them). One or more interface diagrams shall be provided to depict the
interfaces.

\subsection{(Project unique identifier of interface)}

This paragraph (beginning with 3.2) shall identify an interface by
project unique identifier, shall briefly identify the interfacing
entities, and shall be divided into subparagraphs as needed to state the
requirements imposed on one or more of the interfacing entities to
achieve the interface. If the interface characteristics of an entity are
not covered by this IRS but need to be mentioned to specify the
requirements for entities that are, those characteristics shall be
stated as assumptions or as ``When {[}the entity not covered{]} does
this, the {[}entity being specified{]} shall\ldots{},'' rather than as
requirements on the entities not covered by this IRS. This paragraph may
reference other documents (such as data dictionaries, standards for
communication protocols, and standards for user interfaces) in place of
stating the information here. The requirements shall include the
following, as applicable, presented in any order suited to the
requirements, and shall note any differences in these characteristics
from the point of view of the interfacing entities (such as different
expectations about the size, frequency, or other characteristics of data
elements):

\begin{enumerate}
\itemsep1pt\parskip0pt\parsep0pt
\item
  Priority that the interfacing entity(ies) must assign the interface
\item
  Requirements on the type of interface (such as real-time data
  transfer, storage-and-retrieval of data, etc.) to be implemented
\item
  Required characteristics of individual data elements that the
  interfacing entity(ies) must provide, store, send, access, receive,
  etc., such as:

  \begin{enumerate}
  \itemsep1pt\parskip0pt\parsep0pt
  \item
    Names/identifiers

    \begin{enumerate}
    \itemsep1pt\parskip0pt\parsep0pt
    \item
      Project-unique identifier
    \item
      Non-technical (natural-language) name
    \item
      DoD standard data element name
    \item
      Technical name (e.g., variable or field name in code or database)
    \item
      Abbreviation or synonymous names
    \end{enumerate}
  \item
    Data type (alphanumeric, integer, etc.)
  \item
    Size and format (such as length and punctuation of a character
    string)
  \item
    Units of measurement (such as meters, dollars, nanoseconds)
  \item
    Range or enumeration of possible values (such as 0-99)
  \item
    Accuracy (how correct) and precision (number of significant digits)
  \item
    Priority, timing, frequency, volume, sequencing, and other
    constraints, such as whether the data element may be updated and
    whether business rules apply
  \item
    Security and privacy constraints
  \item
    Sources (setting/sending entities) and recipients (using/receiving
    entities)
  \end{enumerate}
\item
  Required characteristics of data element assemblies (records,
  messages, files, arrays, displays, reports, etc.) that the interfacing
  entity(ies) must provide, store, send, access, receive, etc., such as:

  \begin{enumerate}
  \itemsep1pt\parskip0pt\parsep0pt
  \item
    Names/identifiers

    \begin{enumerate}
    \itemsep1pt\parskip0pt\parsep0pt
    \item
      Project-unique identifier
    \item
      Non-technical (natural language) name
    \item
      Technical name (e.g., record or data structure name in code or
      database)
    \item
      Abbreviations or synonymous names
    \end{enumerate}
  \item
    Data elements in the assembly and their structure (number, order,
    grouping)
  \item
    Medium (such as disk) and structure of data elements/assemblies on
    the medium
  \item
    Visual and auditory characteristics of displays and other outputs
    (such as colors, layouts, fonts, icons and other display elements,
    beeps, lights)
  \item
    Relationships among assemblies, such as sorting/access
    characteristics
  \item
    Priority, timing, frequency, volume, sequencing, and other
    constraints, such as whether the assembly may be updated and whether
    business rules apply
  \item
    Security and privacy constraints
  \item
    Sources (setting/sending entities) and recipients (using/receiving
    entities)
  \end{enumerate}
\item
  Required characteristics of communication methods that the interfacing
  entity(ies) must use for the interface, such as:

  \begin{enumerate}
  \itemsep1pt\parskip0pt\parsep0pt
  \item
    Project-unique identifier(s)
  \item
    Communication links/bands/frequencies/media and their
    characteristics
  \item
    Message formatting
  \item
    Flow control (such as sequence numbering and buffer allocation)
  \item
    Data transfer rate, whether periodic/aperiodic, and interval between
    transfers
  \item
    Routing, addressing, and naming conventions
  \item
    Transmission services, including priority and grade
  \item
    Safety/security/privacy considerations, such as encryption, user
    authentication, compartmentalization, and auditing
  \end{enumerate}
\item
  Required characteristics of protocols the interfacing entity(ies) must
  use for the interface, such as:

  \begin{enumerate}
  \itemsep1pt\parskip0pt\parsep0pt
  \item
    Project-unique identifier(s)
  \item
    Priority/layer of the protocol
  \item
    Packeting, including fragmentation and reassembly, routing, and
    addressing
  \item
    Legality checks, error control, and recovery procedures
  \item
    Synchronization, including connection establishment, maintenance,
    termination
  \item
    Status, identification, and any other reporting features
  \end{enumerate}
\item
  Other required characteristics, such as physical compatibility of the
  interfacing entities (dimensions, tolerances, loads, plug
  compatibility, etc.), voltages, etc.
\end{enumerate}

\subsection{3.y Precedence and criticality of requirements}

This paragraph shall be numbered as the last paragraph in Section 3 and
shall specify, if applicable, the order of precedence, criticality, or
assigned weights indicating the relative importance of the requirements
in this specification. Examples include identifying those requirements
deemed critical to safety, to security, or to privacy for purposes of
singling them out for special treatment. If all requirements have equal
weight, this paragraph shall so state.

\section{Qualification provisions}

This section shall define a set of qualification methods and shall
specify, for each requirement in Section 3, the qualification method(s)
to be used to ensure that the requirement has been met. A table may be
used to present this information, or each requirement in Section 3 may
be annotated with the method(s) to be used. Qualification methods may
include:

\begin{enumerate}
\itemsep1pt\parskip0pt\parsep0pt
\item
  Demonstration: The operation of interfacing entities that relies on
  observable functional operation not requiring the use of
  instrumentation, special test equipment, or subsequent analysis.
\item
  Test: The operation of interfacing entities using instrumentation or
  special test equipment to collect data for later analysis.
\item
  Analysis: The processing of accumulated data obtained from other
  qualification methods. Examples are reduction, interpretation, or
  extrapolation of test results.
\item
  Inspection: The visual examination of interfacing entities,
  documentation, etc.
\item
  Special qualification methods: Any special qualification methods for
  the interfacing entities, such as special tools, techniques,
  procedures, facilities, and acceptance limits.
\end{enumerate}

\section{Requirements traceability}

For system-level interfacing entities, this paragraph does not apply.
For each subsystem- or lower-level interfacing entity covered by this
IRS, this paragraph shall contain:

\begin{enumerate}
\itemsep1pt\parskip0pt\parsep0pt
\item
  Traceability from each requirement imposed on the entity in this
  specification to the system (or subsystem, if applicable) requirements
  it addresses. (Alternatively, this traceability may be provided by
  annotating each requirement in Section 3.)

  Note: Each level of
  system refinement may result in requirements not directly traceable to
  higher-level requirements. For example, a system architectural design
  that creates multiple CSCIs may result in requirements about how the
  CSCIs will interface, even though these interfaces are not covered in
  system requirements. Such requirements may be traced to a general
  requirement such as ``system implementation'' or to the system design
  decisions that resulted in their generation.
\item
  Traceability from each system (or subsystem, if applicable)
  requirement that has been allocated to the interfacing entity and that
  affects an interface covered in this specification to the requirements
  in this specification that address it.
\end{enumerate}

\section{Notes}

This section shall contain any general information that aids in
understanding this document (e.g., background information, glossary,
rationale). This section shall include an alphabetical listing of all
acronyms, abbreviations, and their meanings as used in this document and
a list of any terms and definitions needed to understand this document.

\appendix

\section{Appendixes}

Appendixes may be used to provide information published separately for
convenience in document maintenance (e.g., charts, classified data). As
applicable, each appendix shall be referenced in the main body of the
document where the data would normally have been provided. Appendixes
may be bound as separate documents for ease in handling. Appendixes
shall be lettered alphabetically (A, B, etc.).

\end{document}
