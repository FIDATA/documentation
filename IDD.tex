\section{1. Scope.}

This section shall be divided into the following paragraphs.

\subsection{1.1 Identification.}

This paragraph shall contain a full identification of the system(s), the
interfacing entities, and interfaces to which this document applies,
including, as applicable, identification number(s), title(s),
abbreviation(s), version number(s), and release number(s).

\subsection{1.2 System overview.}

This paragraph shall briefly state the purpose of the system(s) and
software to which this document applies. It shall describe the general
nature of the system and software; summarize the history of system
development, operation, and maintenance; identify the project sponsor,
acquirer, user, developer, and support agencies; identify current and
planned operating sites; and list other relevant documents.

\subsection{1.3 Document overview.}

This paragraph shall summarize the purpose and contents of this document
and shall describe any security or privacy considerations associated
with its use.

\section{2. Referenced documents.}

This section shall list the number, title, revision, and date of all
documents referenced in this document. This section shall also identify
the source for all documents not available through normal Government
stocking activities.

\section{3. Interface design.}

This section shall be divided into the following paragraphs to describe
the interface characteristics of one or more systems, subsystems,
configuration items, manual operations, or other system components. If
part or all of the design depends upon system states or modes, this
dependency shall be indicated. If design information falls into more
than one paragraph, it may be presented once and referenced from the
other paragraphs. If part or all of this information is documented
elsewhere, it may be referenced. Design conventions needed to understand
the design shall be presented or referenced.

\subsection{3.1 Interface identification and diagrams.}

For each interface identified in 1.1, this paragraph shall state the
project-unique identifier assigned to the interface and shall identify
the interfacing entities (systems, configuration items, users, etc.) by
name, number, version, and documentation references, as applicable. The
identification shall state which entities have fixed interface
characteristics (and therefore impose interface requirements on
interfacing entities) and which are being developed or modified (thus
having interface requirements imposed on them). One or more interface
diagrams shall be provided, as appropriate, to depict the interfaces.

\subsection{3.x (Project unique identifier of interface).}

This paragraph (beginning with 3.2) shall identify an interface by
project unique identifier, shall briefly identify the interfacing
entities, and shall be divided into subparagraphs as needed to describe
the interface characteristics of one or both of the interfacing
entities. If a given interfacing entity is not covered by this IDD (for
example, an external system) but its interface characteristics need to
be mentioned to describe interfacing entities that are, these
characteristics shall be stated as assumptions or as ``When {[}the
entity not covered{]} does this, {[}the entity that is covered{]} will
\ldots{}.'' This paragraph may reference other documents (such as data
dictionaries, standards for protocols, and standards for user
interfaces) in place of stating the information here. The design
description shall include the following, as applicable, presented in any
order suited to the information to be provided, and shall note any
differences in these characteristics from the point of view of the
interfacing entities (such as different expectations about the size,
frequency, or other characteristics of data elements):

\begin{enumerate}
\itemsep1pt\parskip0pt\parsep0pt
\item
  Priority assigned to the interface by the interfacing entity(ies)
\item
  Type of interface (such as real-time data transfer,
  storage-and-retrieval of data, etc.) to be implemented
\item
  Characteristics of individual data elements that the interfacing
  entity(ies) will provide, store, send, access, receive, etc., such as:

  \begin{enumerate}
  \itemsep1pt\parskip0pt\parsep0pt
  \item
    Names/identifiers

    \begin{enumerate}
    \itemsep1pt\parskip0pt\parsep0pt
    \item
      Project-unique identifier
    \item
      Non-technical (natural-language) name
    \item
      DoD standard data element name
    \item
      Technical name (e.g., variable or field name in code or database)
    \item
      Abbreviation or synonymous names
    \end{enumerate}
  \item
    Data type (alphanumeric, integer, etc.)
  \item
    Size and format (such as length and punctuation of a character
    string)
  \item
    Units of measurement (such as meters, dollars, nanoseconds)
  \item
    Range or enumeration of possible values (such as 0-99)
  \item
    Accuracy (how correct) and precision (number of significant digits)
  \item
    Priority, timing, frequency, volume, sequencing, and other
    constraints, such as whether the data element may be updated and
    whether business rules apply
  \item
    Security and privacy constraints
  \item
    Sources (setting/sending entities) and recipients (using/receiving
    entities)
  \end{enumerate}
\item
  Characteristics of data element assemblies (records, messages, files,
  arrays, displays, reports, etc.) that the interfacing entity(ies) will
  provide, store, send, access, receive, etc., such as:

  \begin{enumerate}
  \itemsep1pt\parskip0pt\parsep0pt
  \item
    Names/identifiers

    \begin{enumerate}
    \itemsep1pt\parskip0pt\parsep0pt
    \item
      Project-unique identifier
    \item
      Non-technical (natural language) name
    \item
      Technical name (e.g., record or data structure name in code or
      database)
    \item
      Abbreviations or synonymous names
    \end{enumerate}
  \item
    Data elements in the assembly and their structure (number, order,
    grouping)
  \item
    Medium (such as disk) and structure of data elements/assemblies on
    the medium
  \item
    Visual and auditory characteristics of displays and other outputs
    (such as colors, layouts, fonts, icons and other display elements,
    beeps, lights)
  \item
    Relationships among assemblies, such as sorting/access
    characteristics
  \item
    Priority, timing, frequency, volume, sequencing, and other
    constraints, such as whether the assembly may be updated and whether
    business rules apply
  \item
    Security and privacy constraints
  \item
    Sources (setting/sending entities) and recipients (using/receiving
    entities)
  \end{enumerate}
\item
  Characteristics of communication methods that the interfacing
  entity(ies) will use for the interface, such as:

  \begin{enumerate}
  \itemsep1pt\parskip0pt\parsep0pt
  \item
    Project-unique identifier(s)
  \item
    Communication links/bands/frequencies/media and their
    characteristics
  \item
    Message formatting
  \item
    Flow control (such as sequence numbering and buffer allocation)
  \item
    Data transfer rate, whether periodic/aperiodic, and interval between
    transfers
  \item
    Routing, addressing, and naming conventions
  \item
    Transmission services, including priority and grade
  \item
    Safety/security/privacy considerations, such as encryption, user
    authentication, compartmentalization, and auditing
  \end{enumerate}
\item
  Characteristics of protocols the interfacing entity(ies) will use for
  the interface, such as:

  \begin{enumerate}
  \itemsep1pt\parskip0pt\parsep0pt
  \item
    Project-unique identifier(s)
  \item
    Priority/layer of the protocol
  \item
    Packeting, including fragmentation and reassembly, routing, and
    addressing
  \item
    Legality checks, error control, and recovery procedures
  \item
    Synchronization, including connection establishment, maintenance,
    termination
  \item
    Status, identification, and any other reporting features
  \end{enumerate}
\item
  Other characteristics, such as physical compatibility of the
  interfacing entity(ies) (dimensions, tolerances, loads, voltages, plug
  compatibility, etc.)
\end{enumerate}

\section{4. Requirements traceability.}

This paragraph shall contain:

\begin{enumerate}
\itemsep1pt\parskip0pt\parsep0pt
\item
  Traceability from each interfacing entity covered by this IDD to the
  system or CSCI requirements addressed by the entity's interface
  design.
\item
  Traceability from each system or CSCI requirement that affects an
  interface covered in this IDD to the interfacing entities that address
  it.
\end{enumerate}

\section{5. Notes.}

This section shall contain any general information that aids in
understanding this document (e.g., background information, glossary,
rationale). This section shall include an alphabetical listing of all
acronyms, abbreviations, and their meanings as used in this document and
a list of any terms and definitions needed to understand this document.

\section{A. Appendixes.}

Appendixes may be used to provide information published separately for
convenience in document maintenance (e.g., charts, classified data). As
applicable, each appendix shall be referenced in the main body of the
document where the data would normally have been provided. Appendixes
may be bound as separate documents for ease in handling. Appendixes
shall be lettered alphabetically (A, B, etc.).
